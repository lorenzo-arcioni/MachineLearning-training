\section{Fondamenti Teorici del KNN}

\subsection{Definizione formale}

Per formalizzare matematicamente l'algoritmo K-Nearest Neighbors (KNN), 
consideriamo un dataset di addestramento \( \mathcal{D} = \{(\mathbf{x}_i, y_i)\}_{i=1}^N \), 
dove \( \mathbf{x}_i \in \mathbb{R}^d \) rappresenta un vettore di lunghezza \( d \) 
(un punto dati con \( d \) caratteristiche) e \( y_i \in \mathbb{R} \) 
(o \( y_i \in \{1, \ldots, C\} \) dove \(C \in \mathbb{N}\), per la classificazione) rappresenta 
un'istanza della variabile target associata.

Assumiamo di aver osservato un insieme di \( n \) punti dati diversi. 
Queste osservazioni sono chiamate dati di addestramento perché li utilizziamo 
per addestrare il nostro modello su come stimare una funzione che ci permette di 
formalizzare la realtà di interesse. Lasciamo che \( x_{ij} \) 
rappresenti il valore del \( j \)-esimo predittore, o input, per l'osservazione \( i \), 
dove \( i = 1, 2, \ldots, n \) e \( j = 1, 2, \ldots, d \). Di conseguenza, sia \( y_i \) 
la variabile di risposta per l'osservazione \( i \). 
I nostri dati di addestramento consistono in \( \{(x_1, y_1), (x_2, y_2), 
\ldots, (x_n, y_n)\} \), dove \( x_i = (x_{i1}, x_{i2}, \ldots, x_{id})^T \).\\

Definiamo ora due funzioni utili a modellare il problema statistico:

\begin{itemize}
    \item La funzione relale \( f \) descrive la relazione tra l'input \( \mathbf{x} \) e la variabile target \( y \). Questa funzione rappresenta il fenomeno o il processo che genera i dati osservati.
    
    \item La funzione stimata \( \hat{f} \) è l'approssimazione di \( f \) 
    ottenuta attraverso un modello statistico o un algoritmo di apprendimento (KNN in questo caso). 
    \( \hat{f} \) cerca di approssimare \( f \) utilizzando i dati di addestramento disponibili per fare previsioni o inferenze su nuovi dati non osservati.
\end{itemize}
Il nostro obiettivo è applicare un metodo di apprendimento statistico 
ai dati di addestramento per stimare la funzione sconosciuta \( f \). 
In altre parole, vogliamo trovare una funzione \( \hat{f} \) tale che 
\( y \approx \hat{f}(\mathbf{x}) \) per qualsiasi osservazione \( (\mathbf{x}, y) \).

I metodi non parametrici, come il KNN, non presuppongono una forma 
specifica per la funzione \( f \). Invece, cercano di stimare \( f \) 
in modo che si avvicini il più possibile ai dati osservati, mantenendo 
un andamento fluido e coerente. Questi approcci offrono un vantaggio 
significativo rispetto ai metodi parametrici poiché non limitano \( f \) 
a una forma predeterminata, permettendo quindi di adattarsi meglio a una 
vasta gamma di possibili configurazioni per \( f \). Nei metodi parametrici, 
invece, c'è il rischio che la forma funzionale $\hat{f}$, utilizzata per stimare \( f \), 
sia molto diversa dalla reale \( f \), portando a modelli che non si adattano 
bene ai dati. Al contrario, i metodi non parametrici evitano completamente 
questo rischio perché non impongono alcuna assunzione sulla forma di \( f \). 
Tuttavia, gli approcci non parametrici hanno uno svantaggio principale: 
essi richiedono un numero elevato di osservazioni per ottenere una stima 
accurata di \( f \), molto più alto rispetto a quanto necessario nei metodi 
parametrici che riducono il problema a un numero limitato di parametri.

\subsection{Metriche di distanza}

Per determinare i \( K \) vicini più prossimi, è necessario definire 
una metrica di distanza \( d(\mathbf{x}, \mathbf{z}) \) tra due vettori (punti dati) 
\( \mathbf{x} \) e \( \mathbf{z} \). Prima di parlare di metriche di distanza, è doveroso fare un'introduzione riguardo alla
norma.\\

Una \textbf{norma} è una funzione che associa a ciascun vettore di uno spazio vettoriale un 
numero reale non negativo e soddisfa specifiche proprietà di compatibilità con la 
struttura dello spazio vettoriale. L'obiettivo principale di una norma è fornire una 
nozione di "lunghezza" dei vettori nello spazio vettoriale considerato. Le proprietà di 
compatibilità con la struttura di spazio vettoriale sono pensate per riflettere alcune 
caratteristiche intuitive della "lunghezza" quando si effettuano operazioni come 
l'addizione di vettori o la moltiplicazione di un vettore per uno scalare.

Una norma su uno spazio vettoriale reale o complesso $\mathcal{X}$, è una funzione:

\begin{align*}
\|\cdot\| \colon \mathcal{X} &\to \mathbb{R}\\
\mathbf{x} &\mapsto \|\mathbf{x}\|
\end{align*}

che verifica le seguenti condizioni:

\begin{itemize}
    \item $\|\mathbf{x}\| \geq 0$, per ogni $\mathbf{x} \in \mathcal{X}$;
    \item $\|\mathbf{x}\| = 0$ se e solo se $\mathbf{x} = 0$;
    \item $\|\lambda \mathbf{x}\| = |\lambda| \|\mathbf{x}\|$, per ogni scalare $\lambda$ (omogeneità);
    \item $\|\mathbf{x} + \mathbf{y}\| \leq \|\mathbf{x}\| + \|\mathbf{y}\|$, per ogni $\mathbf{x}, \mathbf{y} \in \mathcal{X}$ (disuguaglianza triangolare).
\end{itemize}

La coppia $(\mathcal{X}, \|\cdot\|)$ costituisce uno spazio normato.\\

Una metrica di distanza è una funzione che misura la distanza tra due punti in uno spazio. Formalmente, una funzione \( d(\mathbf{x}, \mathbf{z}) \) è una metrica se soddisfa le seguenti proprietà:

\begin{itemize}
    \item \textbf{Non Negatività:} \( d(\mathbf{x}, \mathbf{z}) \geq 0 \) per ogni coppia di punti \( \mathbf{x} \) e \( \mathbf{z} \).
    \item \textbf{Definizione Positiva:} \( d(\mathbf{x}, \mathbf{z}) = 0 \) se e solo se \( \mathbf{x} = \mathbf{z} \).
    \item \textbf{Simmetria:} \( d(\mathbf{x}, \mathbf{z}) = d(\mathbf{z}, \mathbf{x}) \) per ogni coppia di punti \( \mathbf{x} \) e \( \mathbf{z} \).
    \item \textbf{Disuguaglianza Triangolare:} \( d(\mathbf{x}, \mathbf{z}) \leq d(\mathbf{x}, \mathbf{y}) + d(\mathbf{y}, \mathbf{z}) \) per ogni trio di punti \( \mathbf{x} \), \( \mathbf{y} \), e \( \mathbf{z} \).
\end{itemize}

\subsubsection{Distanza Euclidea} 
La distaza Euclidea (chiamata anche norma $L_2$) calcola la distanza diretta tra punti in uno spazio multidimensionale. 
È adatta per dati 
che presentano caratteristiche continue e numeriche con scale e intervalli simili. 
Può anche gestire bene gli outlier e il rumore, poiché dà più peso alle differenze più grandi. 
Tuttavia, può essere influenzata dal "curse of dimensionality", che significa che all'aumentare 
del numero di features, la distanza tra due punti diventa meno significativa e più simile tra loro.
Inoltre, può essere influenzata dall'orientamento e dalla scala delle 
    caratteristiche, poiché assume che tutte le features siano ugualmente 
    importanti e che abbiano tutte una stessa scala.
    \[
    d(\mathbf{x}, \mathbf{z}) = ||\mathbf{x} - \mathbf{z}|| = \sqrt{\sum_{j=1}^d (x_j - z_j)^2}
    \]    

\subsubsection{Distanza Manhattan} La distanza Manhattan (chiamata anche norma $L_1$) 
    è una metrica adatta per dati che presentano caratteristiche discrete e 
    categoriali, poiché non penalizza tanto le piccole differenze 
    quanto la distanza euclidea. Inoltre, gestisce meglio i dati ad alta 
    dimensionalità, poiché è meno sensibile al "curse of dimensionality". 
    Tuttavia, può essere influenzata dall'orientamento e dalla scala delle 
    caratteristiche, poiché assume che tutte le features siano ugualmente 
    importanti e che abbiano tutte una stessa scala.
    \[
    d(\mathbf{x}, \mathbf{z}) = \sum_{j=1}^d |x_j - z_j|
    \]

\subsubsection{Distanza Chebyshev} La distanza Chebyshev (chiamata anche norma $L_{\infty}$) 
    è adatta per dati che presentano caratteristiche discrete e 
    categoriali, poiché non penalizza tanto le piccole differenze 
    quanto la distanza euclidea. Inoltre, gestisce meglio i dati ad alta 
    dimensionalità, poiché è meno sensibile al "curse of dimensionality". 
    Tuttavia, è influenzata dall'orientamento e dalla scala delle 
    caratteristiche, poiché assume che tutte le features siano ugualmente 
    importanti e che abbiano tutte una stessa scala.

    \[
    d(\mathbf{x}, \mathbf{z}) = \sqrt[\infty]{\sum_{j=1}^d |x_j - z_j|^{\infty}} = \max_{j=1}^d |x_j - z_j|
    \]

    Diamo ora una dimostrazione della seconda ugualianza. Siano $\mathbf x$ e $\mathbf z$ due vettori di dimensione $d$.
    
    Sia $\mathbf a = [|x_i - z_i|]^{d}_{i=1}$ il vettore differenza in valore assoluto tra $\mathbf x$ e $\mathbf z$.
    
    Senza perdita di generalità, supponiamo che $\max_{j=1}^d |x_j - z_j| = a_i$.
    
    Allora:
    \[
    \lim_{p \to \infty} (a^p_1 + \ldots + a^p_d)^{1/p} \geq \lim_{p \to \infty} (a^p_i)^{1/p} = \lim_{p \to \infty} a_i = a_i = \max_{j=1}^d |x_j - z_j|
    \]
    
    e:
    \begin{align*}
    \lim_{p \to \infty} (a^p_1 + \ldots + a^p_d)^{1/p} \leq \lim_{p \to \infty} (a^p_i + \ldots + a^p_i)^{1/p} &= \lim_{p \to \infty} (d \cdot a^p_i)^{1/p} \\
    &= \lim_{p \to \infty} a_i \cdot d^{1/p} \\
    &= a_i \cdot \lim_{p \to \infty} d^{1/p} \\
    &= a_i = \max_{j=1}^d |x_j - z_j|
    \end{align*}

    Quindi, per il teorema del confronto, $\lim_{p \to \infty} (a^p_1 + \ldots + a^p_d)^{1/p} = \max_{j=1}^d |x_j - z_j|$. $\square$

\subsubsection{Distanza Minkowski} La distanza Minkowski (chiamata anche norma $L_p$) è
    una generalizzazione delle distanze Euclidea, Manhattan e Chebyshev. 
    È definita da un parametro $p$ che controlla quanto peso viene dato alle differenze più 
    grandi o più piccole tra le coordinate.
    \[
    d(\mathbf{x}, \mathbf{z}) = \left( \sum_{j=1}^d |x_j - z_j|^p \right)^{\frac{1}{p}}
    \]
    dove \( p \) è un parametro positivo che determina la forma della distanza. La distanza 
    Minkowski può essere vista come una famiglia di metriche di distanza che include la 
    distanza Euclidea ($p = 2$), la distanza di Manhattan ($p = 1$) e la distanza di 
    Chebyshev ($p = \infty$), che rappresenta il massimo delle differenze assolute 
    tra le coordinate. La distanza di Minkowski è adatta per dati che presentano tipi misti 
    di caratteristiche, poiché consente di regolare il parametro $p$ per bilanciare 
    l'importanza delle diverse caratteristiche e delle distanze. Tuttavia, può essere 
    computazionalmente costosa e difficile da interpretare, poiché il parametro $p$ può 
    avere effetti diversi su diversi insiemi di dati e problemi.

\subsubsection{Distanza Coseno} La distanza coseno (o similarità coseno) misura 
    l'angolo tra due vettori in uno spazio multidimensionale, piuttosto che la loro distanza diretta. 
    È particolarmente utile per dati in cui l'orientamento dei vettori è più importante della loro 
    magnitudine, come nel caso di dati testuali o vettori di parole (word embeddings). La distanza 
    coseno è calcolata come uno meno il coseno dell'angolo tra i vettori, quindi varia tra 0 e 2, 
    dove 0 indica vettori identici (completamente simili) e 2 indica vettori diametralmente opposti 
    (completamente dissimili).

    \[
d(\mathbf{x}, \mathbf{z}) = 1 - \frac{\mathbf{x} \cdot \mathbf{z}}{||\mathbf{x}|| \cdot ||\mathbf{z}||} = 1 - \frac{\sum_{j=1}^d x_j z_j}{\sqrt{\sum_{j=1}^d x_j^2} \sqrt{\sum_{j=1}^d z_j^2}}
\]

Qui, $\mathbf{x} \cdot \mathbf{z}$ rappresenta il prodotto scalare tra i 
vettori $\mathbf{x}$ e $\mathbf{z}$, mentre $||\mathbf{x}||$ e $||\mathbf{z}||$ sono 
le norme euclidee dei rispettivi vettori.

La distanza coseno è particolarmente robusta rispetto a variazioni di scala nei dati, 
poiché normalizza i vettori prima di calcolare l'angolo tra essi. Questo la rende adatta 
per scenari in cui la lunghezza assoluta (magnitudine) dei vettori è meno rilevante 
rispetto alla direzione, come nell'analisi di 
documenti o nella raccomandazione di oggetti basata su preferenze utente. Tuttavia, può essere 
influenzata dalla presenza di valori nulli o zero nei dati, che possono distorcere il calcolo della similarità.

Diamo ora un'interpretazione intuitiva della distanza coseno. Siano $\mathbf x$ e $\mathbf z$ 
due vettori di dimensione $d$. Il coseno dell'angolo $\alpha$ (l'angolo con ampiezza minore tra i due vettori),
tra $\mathbf x$ e $\mathbf z$ si ottiene, secondo la definizione di coseno, dividendo $||proj_z \mathbf x||$ 
(la lunghezza della 
proiezione del vettore $\mathbf x$ nella direzione del vettore $\mathbf z$) per la norma del vettore $\mathbf x$:

\begin{equation}
\cos(\alpha) = \frac{||proj_z \mathbf x||}{||\mathbf x||}.
\label{eq:cosine}
\end{equation}

Ora non ci rimane che trovare la norma del vettore proiezione. La proiezione $proj_z \mathbf x$ è
un vettore che giace nella direzione di $\mathbf z$, quindi possiamo scriverlo come
$c \cdot \mathbf z$ per qualche numero $c$. Per trovare $c$ possiamo utilizzare il fatto che la 
proiezione $proj_z \mathbf x = c \cdot \mathbf z$ è un vettore che si annulla quando $\mathbf x$ è perpendicolare
a $\mathbf z$. Come vettore perpendicolare a $\mathbf z$ possiamo considerare il vettore $\mathbf x - c \cdot \mathbf z$ 
e moltipliciarlo per $\mathbf z$. Possiamo quindi ricavare il valore $c$ dalla seguente equazione:

\begin{align}
(\mathbf x - c \cdot \mathbf z) \cdot \mathbf z &= \mathbf 0\\
\mathbf x \cdot \mathbf z - c \cdot \mathbf z \cdot \mathbf z &=\mathbf 0\\
\mathbf x \cdot \mathbf z - c ||\mathbf z||^2 &= \mathbf 0\\
c = \frac{\mathbf x \cdot \mathbf z}{||\mathbf z||^2}.
\label{eq:projection}
\end{align}

Ora, mettendo insieme le due equazioni \eqref{eq:cosine} e \eqref{eq:projection}, otteniamo

\begin{align*}
\cos(\alpha) = \frac{||proj_z \mathbf x||}{||\mathbf x||} &= 
\frac{||c \cdot \mathbf z||}{||\mathbf x||} = \frac{c \cdot ||\mathbf z||}{||\mathbf x||}=
\frac{\mathbf x \cdot \mathbf z }{||\mathbf z||^2} \frac{||\mathbf z||}{||\mathbf x||}=
\frac{\mathbf x \cdot \mathbf z}{||\mathbf x|| \cdot ||\mathbf z||}.
\end{align*}

Per ottenere ora un valore positivo che indica la distanza coseno tra due vettori, sottraiamo il coseno,
che ha valori in $[-1, 1]$, ad $1$. In questo modo otteniamo una distanza massima di $2$ e una distanza minima
di $0$.
$\square$

\subsubsection{Distanza di Hamming}
La distanza di Hamming è una metrica comunemente utilizzata per misurare la somiglianza o la dissimiglianza tra due vettori di caratteristiche binarie. È particolarmente utile quando si lavora con dati categorici o binari, dove ogni caratteristica può assumere solo due valori possibili.

Consideriamo due vettori di caratteristiche, $\mathbf{x}$ e $\mathbf{z}$, 
ciascuno composto da $d$ caratteristiche binarie.

La distanza di Hamming tra $\mathbf{x}$ e $\mathbf{z}$, denotata come $d(\mathbf{x}, \mathbf{z})$, può essere calcolata utilizzando la seguente formula:

\[
d(\mathbf{x}, \mathbf{z}) = \sum_{i=1}^{d} (x_i \oplus z_i)
\]

dove $\oplus$ denota l'operazione XOR elemento per elemento. Questo significa che la distanza di Hamming è la somma delle differenze bit a bit tra le caratteristiche corrispondenti dei due vettori.

Per chiarire, l'operazione XOR ($\oplus$) restituisce 1 se $x_i \neq z_i$ e 0 se sono uguali. 
Pertanto, sommando tutti i risultati degli XOR si ottiene il conteggio delle posizioni in cui i due vettori differiscono.

Una variante generalizzata, utilizzabile per vettori di features categoriche (non per forza binarie), 
della distanza di Hamming può essere calcolata utilizzando la seguente formula:

\[
d(\mathbf{x}, \mathbf{z}) = \sum_{i=1}^{d} \mathbf 1_{x_i \neq z_i}
\]

dove $x_i \neq y_i$ indica che la caratteristica $i$ del vettore $\mathbf{x}$ è diversa da quella del vettore $\mathbf{z}$.

La distanza di Hamming è particolarmente conveniente da utilizzare quando le 
caratteristiche dei dati sono binarie (derivate ad esempio da un one-hot encoding) o categoriche 
che possono essere codificate in forma numerica.

\subsubsection{Distanza Mahalanobis}
[...]

\subsection{Algoritmo KNN}

Nel KNN, la variabile target \( \hat{y} \) di un nuovo punto dati \( \hat{\mathbf{x}} \) 
viene determinata come segue:

\begin{enumerate}
    \item Calcolare la distanza, utilizzando la metrica di distanza scelta, tra \( \hat{\mathbf{x}} \) e 
    ogni punto dati \( \mathbf{x}_i \) nel dataset 
    di addestramento \( \mathcal{D} \). Definiamo un nuovo vettore $\mathbf d^{\hat{\mathbf x}}$ che contiene le distanze 
    tra \( \hat{\mathbf{x}} \) e ogni punto di \(\mathcal{D}\):

    $$
    \mathbf d^{\hat{\mathbf x}} = [d(\hat{\mathbf{x}}, \mathbf{x}_i)]^N_{i=1}
    $$

    \item A partire da $\mathbf d^{\hat{\mathbf x}} = [d^{\hat{\mathbf x}}_1, \ldots, d^{\hat{\mathbf x}}_N]$, 
    determiniamo un insieme $\mathcal{N}_K(\hat{\mathbf{x}}) = \{i_1, \ldots, i_k\}$ contenente gli indici
    dei \( K \) vicini più prossimi di \( \hat{\mathbf{x}} \), quindi tale che
    
    $$
    \forall i \in \mathcal{N}_K(\hat{\mathbf{x}}) \ \ \nexists j \in [N] \setminus \mathcal{N}_K(\hat{\mathbf{x}}) : d^{\hat{\mathbf x}}_j < d^{\hat{\mathbf x}}_i.
    $$

    %$$
    %\sigma_{\hat{\mathbf x}} = \arg\sort_{i \in [N]} d^{\hat{\mathbf x}}_i
    %$$
%
    %dove $N$ è la dimensione del dataset $\mathcal D$, $[N]$ è l'insieme $\{1, \ldots, N\}$ e la funzione $\arg\sort$ è una funzione cosi definita:
    %
    %$$
    %\arg\sort_{i \in [N]} v_i = \sigma
    %$$
%
    %dove $v_i$ è un elemento di un vettore $\mathbf v = [v_1, \ldots, v_N]$, $i$ è un indice in $\{1, \ldots, N\}$ e $N$ è il numero di elementi di $\mathbf v$.
    %La funzione $\arg\sort$ restituisce una permutazione $\sigma: \{1, \ldots, N\} \rightarrow \{1, \ldots, N\}$, tale che 
    %$v_{\sigma(i)} \leq v_{\sigma(i+1)} \ \ \forall i \in [N-1]$.
%

    \item Assegnare a \( \hat{y} \):
    
    \begin{itemize}
        
        \item \textbf{Classificazione:} la classe più frequente tra i \( K \) vicini più prossimi. Formalmente,
        \[
    \hat{y} = \arg\max_{c \in \{1, \ldots, C\}} \sum_{i \in \mathcal{N}_K(\hat{\mathbf{x}})} \mathbf{1}_{\{y_i = c\}}
    \]
    dove \( \mathcal{N}_K(\hat{\mathbf{x}}) \) denota l'insieme degli indici dei \( K \) vicini più prossimi 
    di \( \hat{\mathbf{x}} \) e \( \mathbf{1}_{\{y_i = c\}} \) è una funzione indicatrice che vale 1 se \( y_i = c \) e 0 altrimenti.
    

    La funzione di aggregazione più comunemente utilizzata è la moda (la classe più frequente), ma è possibile utilizzarne altre.
    
    \item \textbf{Regresione:} il risultato della funzione di aggregazione (solitamente la media aritmetica) applicata ai valori 
    dei \( K \) vicini più prossimi. Formalmente,
    \[
    \hat{y} = \frac{1}{K} \sum_{i \in \mathcal{N}_K(\hat{\mathbf{x}})} y_i
    \]
    nel caso della media aritmetica.
    Per stabilire il valore da assegnare alla variabile target $\hat{y}$, possono essere utilizzate
    anche altre funzioni di aggregazione, come la mediana, la moda o la media ponderata in base alla distanza dei vicini (minore distanza 
    dal punto = maggiore peso nel calcolo della media). 
    In particolare, in quest'ultimo caso l'algoritmo prende il nome di Weighted KNN. 

    %e formalmente può essere definito come:
%
    %$$
    %\hat{y} = \frac{1}{K} \sum_{i \in \mathcal{N}_K(\hat{\mathbf{x}})} w_i \cdot y_i
    %$$
%
    %dove \( w_i \) è la distanza normalizzata tra \( \hat{\mathbf{x}} \) e il punto \( \mathbf{x}_i \), formalmente,
%
    %$$
    %w_i = \frac{1}{\sum_{i \in \mathcal{N}_K(\hat{\mathbf{x}})} d(\hat{\mathbf{x}}, \mathbf{x}_i)}
    % \frac{d(\hat{\mathbf{x}}, \mathbf{x}_i)}{d(\hat{\mathbf{x}}, \mathbf{x}_{\sigma(i)})}
    %$$

    \end{itemize}
\end{enumerate}

