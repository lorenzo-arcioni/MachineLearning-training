
\section{Introduzione}

\subsection{Panoramica dell'algoritmo K-Nearest Neighbors (KNN)}
L'algoritmo K-Nearest Neighbors (KNN) è un metodo di apprendimento supervisionato utilizzato sia per problemi di classificazione che di regressione. La sua essenza risiede nel principio di vicinanza: gli oggetti simili tendono a trovarsi vicini nello spazio delle caratteristiche. Questa caratteristica rende il KNN intuitivo e semplice da implementare, pur essendo potente in molte applicazioni pratiche.

KNN è un metodo basato sulla prossimità, il che significa che, al momento della previsione per un nuovo dato, l'algoritmo cerca i K punti di addestramento più vicini (i "vicini") e utilizza le loro informazioni per fare la previsione. Per i problemi di classificazione, KNN assegna l'etichetta più comune tra i vicini; per i problemi di regressione, calcola la media dei valori dei vicini.

Un aspetto fondamentale del KNN è la scelta del parametro K, che rappresenta il numero di vicini da considerare. La scelta di K influisce significativamente sulla performance dell'algoritmo: un K troppo piccolo può rendere il modello sensibile al rumore (overfitting), mentre un K troppo grande può diluire la precisione del modello (underfitting).

Un altro elemento critico del KNN è la metrica di distanza utilizzata per determinare la vicinanza tra i punti. Le metriche di distanza più comuni includono la distanza euclidea, la distanza di Manhattan e la distanza di Minkowski, ognuna delle quali ha proprietà diverse che possono influenzare i risultati in base alla natura dei dati.

Nonostante la sua semplicità, KNN presenta alcune sfide, in particolare riguardo alla gestione di grandi dataset e alla sensibilità alla dimensionalità dei dati. Tuttavia, grazie alla sua natura non parametriche e alla facilità di implementazione, rimane un metodo popolare e ampiamente utilizzato in molte applicazioni di machine learning.

\subsection{Importanza e applicazioni del KNN}
\subsection{Obiettivi dell'articolo}