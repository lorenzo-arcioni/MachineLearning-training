\documentclass{article}

% Language setting
% Replace `english' with e.g. `spanish' to change the document language
\usepackage[italian]{babel}

% Set page size and margins
% Replace `letterpaper' with `a4paper' for UK/EU standard size
\usepackage[letterpaper,top=2cm,bottom=2cm,left=3cm,right=3cm,marginparwidth=1.75cm]{geometry}

% Useful packages
\usepackage{amsmath}
\usepackage{amssymb}
\usepackage{graphicx}
\usepackage[colorlinks=true, allcolors=blue]{hyperref}
\usepackage{tikz}
\usepackage{tkz-euclide}
\usepackage{algorithm}%
\usepackage{algorithmicx}%
\usepackage{algpseudocode}%
\usepackage{listings}%
\usepackage{enumitem}
\usepackage{scrextend}
\usepackage{mathtools}

\title{K-Nearest Neighbors}
\author{Lorenzo Arcioni}

\begin{document}
\expandafter\def\expandafter\normalsize\expandafter{
  \setlength\abovedisplayskip{1ex}
  \setlength\belowdisplayskip{4ex}
  \setlength\abovedisplayshortskip{1ex}
  \setlength\belowdisplayshortskip{4ex}
}

\newtheorem{example}{Example}
\newtheorem{definition}{Definition}
\newtheorem{theorem}{Theorem}
\newtheorem{corollary}{Corollary}
\newtheorem{lemma}{Lemma}
\newtheorem{proposition}{Proposition}

\definecolor{def_color}{RGB}{237, 237, 237}
\definecolor{the_color}{RGB}{89, 79, 57}
\definecolor{pro_color_front}{RGB}{71, 93, 107}
\definecolor{pro_color_back}{RGB}{204, 206, 207}
\definecolor{lem_color}{RGB}{204, 206, 207}

\counterwithin{figure}{chapter}

\iffalse
\subsection{Vector Norm}

In linear algebra, the norm of a vector is a measure of its size or magnitude. It provides a quantitative measure of vector size or distance, making it a versatile tool in a wide range of applications.

\subsubsection{Euclidean Norm (2-norm)}

The Euclidean norm of a vector \(\vec{v} = \begin{bmatrix} v_1 \\ v_2 \\ \vdots \\ v_n \end{bmatrix}\) in \(\mathbb{R}^n\) is defined as:

\[
\|\vec{v}\|_2 = \sqrt{v_1^2 + v_2^2 + \ldots + v_n^2}
\]

This norm represents the length of the vector as if it were the hypotenuse of a right-angled triangle in a Cartesian coordinate system.

\subsubsection{Other Norms}

Apart from the Euclidean norm, there are other norms, each with its own characteristics. Some notable examples include:

\begin{itemize}
    \item \textbf{Taxicab Norm (1-norm):}
    \[
    \|\vec{v}\|_1 = |v_1| + |v_2| + \ldots + |v_n|
    \]
    
    \item \textbf{Infinity Norm (\(\infty\)-norm):}
    \[
    \|\vec{v}\|_{\infty} = \max\{|v_1|, |v_2|, \ldots, |v_n|\}
    \]
\end{itemize}

\subsubsection{Properties of Vector Norms}

Vector norms satisfy several key properties, including the triangle inequality:

\[
\|\vec{v} + \vec{w}\| \leq \|\vec{v}\| + \|\vec{w}\|
\]

Understanding vector norms is essential in various applications, such as error analysis, optimization, and signal processing.

\subsubsection{Computational Considerations}

In practice, computing vector norms is often done efficiently using computational tools or libraries. The choice of norm depends on the specific problem at hand, and different norms may be appropriate in different contexts.
\fi

\newcommand{\matrixA}{% 
$\begin{bNiceMatrix}
t_{1,1,1}  & & \Cdots        &t_{1,1,n}\\
\Vdots      &\Ddots     &               &\Vdots \\
            &           &                        \\
t_{1,m,1}  & &  \Cdots       & t_{1,m,n}\\
    \end{bNiceMatrix}$
}   

\newcommand{\matrixI}{% 
    $\begin{bNiceMatrix}
    t_{i,1,1}  & & \Cdots        &t_{i,1,n}\\
    \Vdots      &\Ddots     &               &\Vdots \\
    &           &                           &\\
    t_{i,m,1}  & \Cdots&               & t_{i,m,n}\\
    \end{bNiceMatrix}$
}

\newcommand{\matrixB}{% 
    $\begin{bNiceMatrix}
    t_{k,1,1}  & & \Cdots        &t_{k,1,n}\\
    \Vdots      &\Ddots     &               &\Vdots \\
    &           &                           &\\
    t_{k,m,1}  & \Cdots&               & t_{k,m,n}\\
    \end{bNiceMatrix}$
}
%%%%%%%%%%%%%%%%%%%%%%%%%%%%%%%%%SCALAR-TENSOR-PRODUCT%%%%%%%%%%%%%%%%%%%%%%%%%%%%%%%%%
\newcommand{\matrixC}{% 
    $\begin{bNiceMatrix}
    c \cdot a_{11}^{k}  & & \Cdots        &c \cdot a_{1n}^{k}\\
    \Vdots      &\Ddots     &               &\Vdots \\
    &           &                           &\\
    c \cdot a_{m1}^{k}  & \Cdots&               &c \cdot a_{mn}^{k}\\
    \end{bNiceMatrix}$
}

\newcommand{\matrixD}{% 
    $\begin{bNiceMatrix}
    c \cdot a_{11}^{i}  & & \Cdots        &c \cdot a_{1n}^{i}\\
    \Vdots      &\Ddots     &               &\Vdots \\
    &           &                           &\\
    c \cdot a_{m1}^{i}  & \Cdots&               &c \cdot a_{mn}^{i}\\
    \end{bNiceMatrix}$
}

\newcommand{\matrixE}{% 
    $\begin{bNiceMatrix}
    c \cdot a_{11}^{k}  & & \Cdots        &c \cdot a_{1n}^{k}\\
    \Vdots      &\Ddots     &               &\Vdots \\
    &           &                           &\\
    c \cdot a_{m1}^{k}  & \Cdots&               &c \cdot a_{mn}^{k}\\
    \end{bNiceMatrix}$
}

\newcommand{\notimplies}{\hphantom{0}\!\!\not\!\!\!\!\implies}


\maketitle

\begin{abstract}
    In questo articolo, presentiamo un'analisi approfondita dell'algoritmo K-Nearest Neighbors (KNN), 
    esaminandolo sia dal punto di vista teorico che pratico. L'algoritmo KNN è un metodo di 
    apprendimento supervisionato utilizzato per la classificazione e la regressione, basato sul 
    principio che oggetti simili sono vicini nello spazio delle caratteristiche. Iniziamo con una 
    descrizione dettagliata dei fondamenti teorici del KNN, compresa la definizione formale, 
    i criteri di scelta del parametro K e le metriche di distanza utilizzate per determinare la 
    vicinanza tra i dati. Successivamente, esploriamo le sue proprietà matematiche e discutiamo l'impatto 
    della dimensionalità dei dati e del rumore sulla sua performance. Attraverso un'analisi empirica, 
    confrontiamo l'efficacia del KNN con altri algoritmi di machine learning, utilizzando dataset 
    standard. Infine, esaminiamo le tecniche di ottimizzazione e miglioramento del KNN, come 
    la normalizzazione dei dati e l'uso di pesi nei vicini, per aumentare la precisione e l'efficienza 
    computazionale. Questo studio offre una visione completa del KNN, evidenziando i suoi punti di forza, 
    le sue limitazioni e le situazioni in cui è più adatto. 
\end{abstract}

\tableofcontents
\documentclass{article}

% Language setting
% Replace `english' with e.g. `spanish' to change the document language
\usepackage[italian]{babel}

% Set page size and margins
% Replace `letterpaper' with `a4paper' for UK/EU standard size
\usepackage[letterpaper,top=2cm,bottom=2cm,left=3cm,right=3cm,marginparwidth=1.75cm]{geometry}

% Useful packages
\usepackage{amsmath}
\usepackage{amssymb}
\usepackage{graphicx}
\usepackage[colorlinks=true, allcolors=blue]{hyperref}
\usepackage{tikz}
\usepackage{tkz-euclide}
\usepackage{algorithm}%
\usepackage{algorithmicx}%
\usepackage{algpseudocode}%
\usepackage{listings}%
\usepackage{enumitem}
\usepackage{scrextend}
\usepackage{mathtools}

\title{K-Nearest Neighbors}
\author{Lorenzo Arcioni}

\begin{document}
\expandafter\def\expandafter\normalsize\expandafter{
  \setlength\abovedisplayskip{1ex}
  \setlength\belowdisplayskip{4ex}
  \setlength\abovedisplayshortskip{1ex}
  \setlength\belowdisplayshortskip{4ex}
}

\newtheorem{example}{Example}
\newtheorem{definition}{Definition}
\newtheorem{theorem}{Theorem}
\newtheorem{corollary}{Corollary}
\newtheorem{lemma}{Lemma}
\newtheorem{proposition}{Proposition}

\definecolor{def_color}{RGB}{237, 237, 237}
\definecolor{the_color}{RGB}{89, 79, 57}
\definecolor{pro_color_front}{RGB}{71, 93, 107}
\definecolor{pro_color_back}{RGB}{204, 206, 207}
\definecolor{lem_color}{RGB}{204, 206, 207}

\counterwithin{figure}{chapter}

\iffalse
\subsection{Vector Norm}

In linear algebra, the norm of a vector is a measure of its size or magnitude. It provides a quantitative measure of vector size or distance, making it a versatile tool in a wide range of applications.

\subsubsection{Euclidean Norm (2-norm)}

The Euclidean norm of a vector \(\vec{v} = \begin{bmatrix} v_1 \\ v_2 \\ \vdots \\ v_n \end{bmatrix}\) in \(\mathbb{R}^n\) is defined as:

\[
\|\vec{v}\|_2 = \sqrt{v_1^2 + v_2^2 + \ldots + v_n^2}
\]

This norm represents the length of the vector as if it were the hypotenuse of a right-angled triangle in a Cartesian coordinate system.

\subsubsection{Other Norms}

Apart from the Euclidean norm, there are other norms, each with its own characteristics. Some notable examples include:

\begin{itemize}
    \item \textbf{Taxicab Norm (1-norm):}
    \[
    \|\vec{v}\|_1 = |v_1| + |v_2| + \ldots + |v_n|
    \]
    
    \item \textbf{Infinity Norm (\(\infty\)-norm):}
    \[
    \|\vec{v}\|_{\infty} = \max\{|v_1|, |v_2|, \ldots, |v_n|\}
    \]
\end{itemize}

\subsubsection{Properties of Vector Norms}

Vector norms satisfy several key properties, including the triangle inequality:

\[
\|\vec{v} + \vec{w}\| \leq \|\vec{v}\| + \|\vec{w}\|
\]

Understanding vector norms is essential in various applications, such as error analysis, optimization, and signal processing.

\subsubsection{Computational Considerations}

In practice, computing vector norms is often done efficiently using computational tools or libraries. The choice of norm depends on the specific problem at hand, and different norms may be appropriate in different contexts.
\fi

\newcommand{\matrixA}{% 
$\begin{bNiceMatrix}
t_{1,1,1}  & & \Cdots        &t_{1,1,n}\\
\Vdots      &\Ddots     &               &\Vdots \\
            &           &                        \\
t_{1,m,1}  & &  \Cdots       & t_{1,m,n}\\
    \end{bNiceMatrix}$
}   

\newcommand{\matrixI}{% 
    $\begin{bNiceMatrix}
    t_{i,1,1}  & & \Cdots        &t_{i,1,n}\\
    \Vdots      &\Ddots     &               &\Vdots \\
    &           &                           &\\
    t_{i,m,1}  & \Cdots&               & t_{i,m,n}\\
    \end{bNiceMatrix}$
}

\newcommand{\matrixB}{% 
    $\begin{bNiceMatrix}
    t_{k,1,1}  & & \Cdots        &t_{k,1,n}\\
    \Vdots      &\Ddots     &               &\Vdots \\
    &           &                           &\\
    t_{k,m,1}  & \Cdots&               & t_{k,m,n}\\
    \end{bNiceMatrix}$
}
%%%%%%%%%%%%%%%%%%%%%%%%%%%%%%%%%SCALAR-TENSOR-PRODUCT%%%%%%%%%%%%%%%%%%%%%%%%%%%%%%%%%
\newcommand{\matrixC}{% 
    $\begin{bNiceMatrix}
    c \cdot a_{11}^{k}  & & \Cdots        &c \cdot a_{1n}^{k}\\
    \Vdots      &\Ddots     &               &\Vdots \\
    &           &                           &\\
    c \cdot a_{m1}^{k}  & \Cdots&               &c \cdot a_{mn}^{k}\\
    \end{bNiceMatrix}$
}

\newcommand{\matrixD}{% 
    $\begin{bNiceMatrix}
    c \cdot a_{11}^{i}  & & \Cdots        &c \cdot a_{1n}^{i}\\
    \Vdots      &\Ddots     &               &\Vdots \\
    &           &                           &\\
    c \cdot a_{m1}^{i}  & \Cdots&               &c \cdot a_{mn}^{i}\\
    \end{bNiceMatrix}$
}

\newcommand{\matrixE}{% 
    $\begin{bNiceMatrix}
    c \cdot a_{11}^{k}  & & \Cdots        &c \cdot a_{1n}^{k}\\
    \Vdots      &\Ddots     &               &\Vdots \\
    &           &                           &\\
    c \cdot a_{m1}^{k}  & \Cdots&               &c \cdot a_{mn}^{k}\\
    \end{bNiceMatrix}$
}

\newcommand{\notimplies}{\hphantom{0}\!\!\not\!\!\!\!\implies}


\maketitle

\begin{abstract}
    In questo articolo, presentiamo un'analisi approfondita dell'algoritmo K-Nearest Neighbors (KNN), 
    esaminandolo sia dal punto di vista teorico che pratico. L'algoritmo KNN è un metodo di 
    apprendimento supervisionato utilizzato per la classificazione e la regressione, basato sul 
    principio che oggetti simili sono vicini nello spazio delle caratteristiche. Iniziamo con una 
    descrizione dettagliata dei fondamenti teorici del KNN, compresa la definizione formale, 
    i criteri di scelta del parametro K e le metriche di distanza utilizzate per determinare la 
    vicinanza tra i dati. Successivamente, esploriamo le sue proprietà matematiche e discutiamo l'impatto 
    della dimensionalità dei dati e del rumore sulla sua performance. Attraverso un'analisi empirica, 
    confrontiamo l'efficacia del KNN con altri algoritmi di machine learning, utilizzando dataset 
    standard. Infine, esaminiamo le tecniche di ottimizzazione e miglioramento del KNN, come 
    la normalizzazione dei dati e l'uso di pesi nei vicini, per aumentare la precisione e l'efficienza 
    computazionale. Questo studio offre una visione completa del KNN, evidenziando i suoi punti di forza, 
    le sue limitazioni e le situazioni in cui è più adatto. 
\end{abstract}

\tableofcontents
\documentclass{article}

% Language setting
% Replace `english' with e.g. `spanish' to change the document language
\usepackage[italian]{babel}

% Set page size and margins
% Replace `letterpaper' with `a4paper' for UK/EU standard size
\usepackage[letterpaper,top=2cm,bottom=2cm,left=3cm,right=3cm,marginparwidth=1.75cm]{geometry}

% Useful packages
\usepackage{amsmath}
\usepackage{amssymb}
\usepackage{graphicx}
\usepackage[colorlinks=true, allcolors=blue]{hyperref}
\usepackage{tikz}
\usepackage{tkz-euclide}
\usepackage{algorithm}%
\usepackage{algorithmicx}%
\usepackage{algpseudocode}%
\usepackage{listings}%
\usepackage{enumitem}
\usepackage{scrextend}
\usepackage{mathtools}

\title{K-Nearest Neighbors}
\author{Lorenzo Arcioni}

\begin{document}
\expandafter\def\expandafter\normalsize\expandafter{
  \setlength\abovedisplayskip{1ex}
  \setlength\belowdisplayskip{4ex}
  \setlength\abovedisplayshortskip{1ex}
  \setlength\belowdisplayshortskip{4ex}
}

\newtheorem{example}{Example}
\newtheorem{definition}{Definition}
\newtheorem{theorem}{Theorem}
\newtheorem{corollary}{Corollary}
\newtheorem{lemma}{Lemma}
\newtheorem{proposition}{Proposition}

\definecolor{def_color}{RGB}{237, 237, 237}
\definecolor{the_color}{RGB}{89, 79, 57}
\definecolor{pro_color_front}{RGB}{71, 93, 107}
\definecolor{pro_color_back}{RGB}{204, 206, 207}
\definecolor{lem_color}{RGB}{204, 206, 207}

\counterwithin{figure}{chapter}

\iffalse
\subsection{Vector Norm}

In linear algebra, the norm of a vector is a measure of its size or magnitude. It provides a quantitative measure of vector size or distance, making it a versatile tool in a wide range of applications.

\subsubsection{Euclidean Norm (2-norm)}

The Euclidean norm of a vector \(\vec{v} = \begin{bmatrix} v_1 \\ v_2 \\ \vdots \\ v_n \end{bmatrix}\) in \(\mathbb{R}^n\) is defined as:

\[
\|\vec{v}\|_2 = \sqrt{v_1^2 + v_2^2 + \ldots + v_n^2}
\]

This norm represents the length of the vector as if it were the hypotenuse of a right-angled triangle in a Cartesian coordinate system.

\subsubsection{Other Norms}

Apart from the Euclidean norm, there are other norms, each with its own characteristics. Some notable examples include:

\begin{itemize}
    \item \textbf{Taxicab Norm (1-norm):}
    \[
    \|\vec{v}\|_1 = |v_1| + |v_2| + \ldots + |v_n|
    \]
    
    \item \textbf{Infinity Norm (\(\infty\)-norm):}
    \[
    \|\vec{v}\|_{\infty} = \max\{|v_1|, |v_2|, \ldots, |v_n|\}
    \]
\end{itemize}

\subsubsection{Properties of Vector Norms}

Vector norms satisfy several key properties, including the triangle inequality:

\[
\|\vec{v} + \vec{w}\| \leq \|\vec{v}\| + \|\vec{w}\|
\]

Understanding vector norms is essential in various applications, such as error analysis, optimization, and signal processing.

\subsubsection{Computational Considerations}

In practice, computing vector norms is often done efficiently using computational tools or libraries. The choice of norm depends on the specific problem at hand, and different norms may be appropriate in different contexts.
\fi

\newcommand{\matrixA}{% 
$\begin{bNiceMatrix}
t_{1,1,1}  & & \Cdots        &t_{1,1,n}\\
\Vdots      &\Ddots     &               &\Vdots \\
            &           &                        \\
t_{1,m,1}  & &  \Cdots       & t_{1,m,n}\\
    \end{bNiceMatrix}$
}   

\newcommand{\matrixI}{% 
    $\begin{bNiceMatrix}
    t_{i,1,1}  & & \Cdots        &t_{i,1,n}\\
    \Vdots      &\Ddots     &               &\Vdots \\
    &           &                           &\\
    t_{i,m,1}  & \Cdots&               & t_{i,m,n}\\
    \end{bNiceMatrix}$
}

\newcommand{\matrixB}{% 
    $\begin{bNiceMatrix}
    t_{k,1,1}  & & \Cdots        &t_{k,1,n}\\
    \Vdots      &\Ddots     &               &\Vdots \\
    &           &                           &\\
    t_{k,m,1}  & \Cdots&               & t_{k,m,n}\\
    \end{bNiceMatrix}$
}
%%%%%%%%%%%%%%%%%%%%%%%%%%%%%%%%%SCALAR-TENSOR-PRODUCT%%%%%%%%%%%%%%%%%%%%%%%%%%%%%%%%%
\newcommand{\matrixC}{% 
    $\begin{bNiceMatrix}
    c \cdot a_{11}^{k}  & & \Cdots        &c \cdot a_{1n}^{k}\\
    \Vdots      &\Ddots     &               &\Vdots \\
    &           &                           &\\
    c \cdot a_{m1}^{k}  & \Cdots&               &c \cdot a_{mn}^{k}\\
    \end{bNiceMatrix}$
}

\newcommand{\matrixD}{% 
    $\begin{bNiceMatrix}
    c \cdot a_{11}^{i}  & & \Cdots        &c \cdot a_{1n}^{i}\\
    \Vdots      &\Ddots     &               &\Vdots \\
    &           &                           &\\
    c \cdot a_{m1}^{i}  & \Cdots&               &c \cdot a_{mn}^{i}\\
    \end{bNiceMatrix}$
}

\newcommand{\matrixE}{% 
    $\begin{bNiceMatrix}
    c \cdot a_{11}^{k}  & & \Cdots        &c \cdot a_{1n}^{k}\\
    \Vdots      &\Ddots     &               &\Vdots \\
    &           &                           &\\
    c \cdot a_{m1}^{k}  & \Cdots&               &c \cdot a_{mn}^{k}\\
    \end{bNiceMatrix}$
}

\newcommand{\notimplies}{\hphantom{0}\!\!\not\!\!\!\!\implies}


\maketitle

\begin{abstract}
    In questo articolo, presentiamo un'analisi approfondita dell'algoritmo K-Nearest Neighbors (KNN), 
    esaminandolo sia dal punto di vista teorico che pratico. L'algoritmo KNN è un metodo di 
    apprendimento supervisionato utilizzato per la classificazione e la regressione, basato sul 
    principio che oggetti simili sono vicini nello spazio delle caratteristiche. Iniziamo con una 
    descrizione dettagliata dei fondamenti teorici del KNN, compresa la definizione formale, 
    i criteri di scelta del parametro K e le metriche di distanza utilizzate per determinare la 
    vicinanza tra i dati. Successivamente, esploriamo le sue proprietà matematiche e discutiamo l'impatto 
    della dimensionalità dei dati e del rumore sulla sua performance. Attraverso un'analisi empirica, 
    confrontiamo l'efficacia del KNN con altri algoritmi di machine learning, utilizzando dataset 
    standard. Infine, esaminiamo le tecniche di ottimizzazione e miglioramento del KNN, come 
    la normalizzazione dei dati e l'uso di pesi nei vicini, per aumentare la precisione e l'efficienza 
    computazionale. Questo studio offre una visione completa del KNN, evidenziando i suoi punti di forza, 
    le sue limitazioni e le situazioni in cui è più adatto. 
\end{abstract}

\tableofcontents
\documentclass{article}

% Language setting
% Replace `english' with e.g. `spanish' to change the document language
\usepackage[italian]{babel}

% Set page size and margins
% Replace `letterpaper' with `a4paper' for UK/EU standard size
\usepackage[letterpaper,top=2cm,bottom=2cm,left=3cm,right=3cm,marginparwidth=1.75cm]{geometry}

% Useful packages
\usepackage{amsmath}
\usepackage{amssymb}
\usepackage{graphicx}
\usepackage[colorlinks=true, allcolors=blue]{hyperref}
\usepackage{tikz}
\usepackage{tkz-euclide}
\usepackage{algorithm}%
\usepackage{algorithmicx}%
\usepackage{algpseudocode}%
\usepackage{listings}%
\usepackage{enumitem}
\usepackage{scrextend}
\usepackage{mathtools}

\title{K-Nearest Neighbors}
\author{Lorenzo Arcioni}

\begin{document}
\input{my_definitions}

\maketitle

\begin{abstract}
    In questo articolo, presentiamo un'analisi approfondita dell'algoritmo K-Nearest Neighbors (KNN), 
    esaminandolo sia dal punto di vista teorico che pratico. L'algoritmo KNN è un metodo di 
    apprendimento supervisionato utilizzato per la classificazione e la regressione, basato sul 
    principio che oggetti simili sono vicini nello spazio delle caratteristiche. Iniziamo con una 
    descrizione dettagliata dei fondamenti teorici del KNN, compresa la definizione formale, 
    i criteri di scelta del parametro K e le metriche di distanza utilizzate per determinare la 
    vicinanza tra i dati. Successivamente, esploriamo le sue proprietà matematiche e discutiamo l'impatto 
    della dimensionalità dei dati e del rumore sulla sua performance. Attraverso un'analisi empirica, 
    confrontiamo l'efficacia del KNN con altri algoritmi di machine learning, utilizzando dataset 
    standard. Infine, esaminiamo le tecniche di ottimizzazione e miglioramento del KNN, come 
    la normalizzazione dei dati e l'uso di pesi nei vicini, per aumentare la precisione e l'efficienza 
    computazionale. Questo studio offre una visione completa del KNN, evidenziando i suoi punti di forza, 
    le sue limitazioni e le situazioni in cui è più adatto. 
\end{abstract}

\tableofcontents
\input{main.toc}

\input{Chapters/01_Introduzione.tex}
\input{Chapters/02_Fondamenti_teorici.tex}
\input{Chapters/03_Analisi_Teorica.tex}
\input{Chapters/04_Ottimizzazioni.tex}

\nocite{*}
\bibliographystyle{unsrt}
\bibliography{sample.bib}

\end{document}


\section{Introduzione}

\subsection{Panoramica dell'algoritmo K-Nearest Neighbors (KNN)}
L'algoritmo K-Nearest Neighbors (KNN) rappresenta un pilastro fondamentale 
nell'ambito dell'apprendimento automatico supervisionato, apprezzato per la 
sua semplicità concettuale e la sua efficacia in una vasta gamma di applicazioni. 
La sua filosofia si basa sul principio intuitivo che oggetti simili tendono a 
raggrupparsi nello stesso spazio delle caratteristiche. Questo approccio non 
parametrico permette a KNN di adattarsi a strutture dati complesse e 
a relazioni non lineari, senza fare assunzioni rigide sulla distribuzione dei dati.

\subsection{Funzionamento di KNN}

KNN opera determinando le etichette di classificazione o i valori di 
regressione per un nuovo punto, basandosi sulla vicinanza ai punti di addestramento. 
Il parametro chiave di KNN è $K$, che rappresenta il numero di "vicini" più prossimi 
da considerare durante la fase di predizione. Quando un nuovo dato deve essere 
classificato, KNN calcola la distanza (come vedremo, esistono varie tipologie di distanze) 
tra il dato da classificare 
e tutti i punti di addestramento, quindi seleziona i $K$ punti più vicini. La classe o 
il valore di regressione del nuovo dato è determinato dalla classe maggiormente rappresentata
o dalla media dei valori nei punti vicini, rispettivamente.

\subsection{Potenzialità di KNN}

KNN trova applicazione in numerosi settori grazie alla sua flessibilità e facilità di 
implementazione. Nei sistemi di raccomandazione, ad esempio, può suggerire prodotti o 
contenuti simili a quelli preferiti dall'utente sulla base dei gusti di altri utenti simili 
(vicini). Nell'analisi di immagini e nel riconoscimento di pattern, KNN può classificare 
nuove immagini confrontandole con esempi già noti. In ambito medico, può supportare la diagnosi 
confrontando i sintomi del paziente con casi storici simili.

\subsection{Caratteristiche e Limitazioni}

Una delle caratteristiche distintive di KNN è la sua interpretabilità. Le decisioni di 
classificazione o regressione si basano direttamente sulla vicinanza tra i dati, rendendo 
il processo decisionale trasparente e facilmente comprensibile. Tuttavia, KNN può essere 
sensibile al rumore nei dati e alla presenza di feature non rilevanti, il che può influenzare 
negativamente le previsioni. Inoltre, gestire grandi dataset con KNN può essere computazionalmente 
oneroso, poiché richiede il calcolo delle distanze tra il nuovo dato e tutti i punti di addestramento; 
sopratutto quando si ha a che fare con dataset dimensionalmente complessi.

\subsection{Definizione e concetto di base}

\subsubsection{Dataset, feature e variabile target}

Un dataset in machine learning è una collezione di dati organizzati in un formato strutturato. Ogni riga del dataset rappresenta un'osservazione, mentre ogni colonna rappresenta una caratteristica (feature) o la variabile target (etichetta). Le feature sono attributi che descrivono le osservazioni e possono essere di diversi tipi, come numeriche, categoriche o binarie. La variabile target è ciò che vogliamo predire utilizzando le feature.

Ad esempio, in un dataset per la predizione del prezzo delle case, le feature potrebbero includere la superficie della casa, il numero di camere, la posizione, l'anno di costruzione, ecc. La variabile target sarebbe il prezzo della casa.

\subsubsection{Problema di classificazione}

Consideriamo un esempio reale: la diagnosi precoce di malattie cardiache. Questo è un problema non banale che richiede l'uso del machine learning per essere risolto efficacemente. Il dataset potrebbe includere pazienti con varie caratteristiche cliniche misurate durante esami medici. Le feature potrebbero includere età, genere, pressione sanguigna, livelli di colesterolo, frequenza cardiaca massima, risultati di elettrocardiogrammi, e altre misure cliniche rilevanti. La variabile target sarebbe una variabile binaria che indica la presenza o l'assenza di una malattia cardiaca.

L'algoritmo KNN può essere utilizzato per classificare un nuovo paziente come "a rischio" o "non a rischio" di malattia cardiaca basandosi sui dati storici di altri pazienti. Quando un nuovo paziente entra per una valutazione, KNN calcola le distanze tra le caratteristiche cliniche del nuovo paziente e quelle dei pazienti nel dataset di addestramento. Seleziona i \( K \) pazienti più simili (i vicini più prossimi) e determina la classe del nuovo paziente in base alla maggioranza delle classi dei vicini.

Per esempio, se \( K = 5 \) e tra i 5 pazienti più vicini al nuovo paziente 3 hanno una malattia cardiaca e 2 no, KNN predice che il nuovo paziente è "a rischio" di malattia cardiaca. Questa previsione può aiutare i medici a prendere decisioni informate riguardo ulteriori test o trattamenti, dimostrando l'importanza e l'utilità del Machine Learning in contesti medici critici.

\subsubsection{Problema di regressione}

Un esempio di problema di regressione risolvibile con l'algoritmo K-Nearest Neighbors (KNN) è la previsione del valore di mercato delle proprietà immobiliari in una città. Questo problema richiede un approccio che si affida al machine learning per ottenere stime accurate e affidabili, data la moltitudine di fattori che influenzano i prezzi delle case.

Consideriamo un dataset che include informazioni dettagliate sulle proprietà immobiliari di una città. Le feature possono includere:

\begin{itemize}
    \item Superficie della proprietà (in metri quadrati)
    \item Numero di camere da letto
    \item Numero di bagni
    \item Anno di costruzione
    \item Distanza dai servizi principali (scuole, ospedali, trasporti pubblici)
    \item Valutazioni della qualità del quartiere
    \item Prezzi recenti delle proprietà vicine
\end{itemize}

La variabile target in questo caso è il prezzo di vendita della proprietà.

Quando si vuole stimare il valore di una nuova proprietà, l'algoritmo KNN calcola la distanza tra le caratteristiche della nuova proprietà e quelle delle proprietà nel dataset di addestramento. Utilizzando una metrica di distanza (come la distanza euclidea), KNN identifica i \( K \) immobili più simili. 

Ad esempio, se \( K = 5 \), KNN selezionerà le cinque proprietà più vicine alla nuova proprietà in termini di caratteristiche. Il prezzo stimato per la nuova proprietà sarà la media dei prezzi delle cinque proprietà più vicine.

\[
\hat{y} = \frac{1}{K} \sum_{i \in \mathcal{N}_K(\mathbf{x})} y_i
\]

Dove \( \hat{y} \) è il prezzo stimato della nuova proprietà, \( K \) è il numero di vicini considerati, \( \mathcal{N}_K(\mathbf{x}) \) rappresenta l'insieme dei \( K \) vicini più prossimi e \( y_i \) è il prezzo di una delle proprietà vicine.

Questo approccio di regressione basato su KNN è particolarmente utile perché tiene conto della località spaziale e delle caratteristiche specifiche delle proprietà immobiliari. Inoltre, permette di adattarsi a variazioni non lineari e complesse nei dati, che sono comuni nel mercato immobiliare. La previsione accurata dei prezzi immobiliari è fondamentale per acquirenti, venditori, agenti immobiliari e investitori, rendendo KNN uno strumento prezioso in questo contesto.


\subsection{Sfide e Ottimizzazioni}

La "maledizione della dimensionalità" è una delle sfide principali di KNN, in quanto la 
performance dell'algoritmo può decadere significativamente con l'aumento della dimensionalità 
dei dati. Per mitigare questo problema, sono state sviluppate tecniche come la riduzione della 
dimensionalità e l'uso di strutture dati specializzate (come KD-Trees) per accelerare il calcolo 
delle distanze.

\subsection{Applicazioni Avanzate e Ricerca}

La ricerca attuale su KNN si concentra sulla sua integrazione con tecniche avanzate di 
machine learning, come l'apprendimento semi-supervisionato e il trasferimento di conoscenza, 
per migliorare ulteriormente la sua robustezza e la sua capacità predittiva in scenari complessi.

Concludendo questa introduzione, KNN rimane una scelta potente e versatile per molte applicazioni di machine learning grazie alla sua semplicità, flessibilità e interpretabilità. Nonostante le sfide associate, continua a essere ampiamente utilizzato come punto di partenza per problemi di classificazione e regressione, offrendo risultati affidabili e interpretazioni chiare in vari contesti applicativi.

\subsection{Obiettivi dell'articolo}

L'obiettivo principale di questo articolo è fornire una comprensione completa e dettagliata 
dell'algoritmo K-Nearest Neighbors (KNN) attraverso un'analisi sia teorica che pratica. 
In primo luogo, l'articolo presenterà i fondamenti teorici di KNN, spiegando il funzionamento 
dell'algoritmo, le metriche di distanza utilizzate (come la distanza euclidea e la distanza di Manhattan) 
e l'importanza della scelta del parametro $K$. Verranno, inoltre, discusse le implicazioni di queste scelte 
sulla performance dell'algoritmo.

In secondo luogo, verranno esaminate le proprietà matematiche di KNN, inclusa la complessità 
computazionale e le sfide legate alla "maledizione della dimensionalità". Saranno analizzati i 
trade-off tra bias e varianza per comprendere come ottimizzare le prestazioni di questo algoritmo.

In terzo luogo, l'articolo fornirà un'analisi empirica, confrontando KNN con altri algoritmi 
di machine learning su dataset standard. Questo confronto aiuterà a evidenziare i punti di forza 
e le limitazioni di KNN in scenari pratici.

Infine, l'articolo esplorerà le tecniche di ottimizzazione e miglioramento di KNN, 
come la normalizzazione dei dati, l'uso di KNN pesato e l'implementazione di strutture 
dati efficienti come KD-Trees. 

In sintesi, l'articolo mira a offrire una visione completa e 
dettagliata dell'algoritmo KNN.
\section{Fondamenti Teorici del KNN}
\subsection{Definizione e concetto di base}
\subsection{Definizione matematica formale}
\subsection{Scelta del parametro K}
\subsection{Metriche di distanza}
\subsubsection{Distanza Euclidea}
\subsubsection{Distanza di Manhattan}
\subsubsection{Distanza di Minkowski}
\subsubsection{Altre metriche di distanza}

\section{Proprietà Matematiche e Analisi Teorica}
\subsection{La maledizione della dimensionalità}
\subsection{Complessità computazionale}
\subsection{Trade-off bias-varianza nel KNN}
\subsection{Interpretazione probabilistica del KNN}
\subsection{Comportamento asintotico e convergenza}
\section{Analisi Teorica}
\subsection{La maledizione della dimensionalità}
\subsection{Complessità computazionale}
\subsection{Trade-off bias-varianza nel KNN}
\subsection{Scelta di \( K \)}

Nel contesto del KNN, il (iper)parametro $K$ gioca un ruolo cruciale nel determinare il trade-off tra bias e varianza:

\subsubsection{Piccoli Valori di $K$}

Quando $K$ è piccolo (ad esempio, $K=1$), il modello tende a seguire 
molto da vicino i dati di addestramento. Questo può portare a un basso bias, 
poiché il modello è molto flessibile e può adattarsi alle particolarità dei dati 
di addestramento. Tuttavia, questo porta a una elevata varianza, poiché il modello 
è sensibile al rumore nei dati. In altre parole, un valore di $K$ troppo piccolo può 
causare overfitting.

\subsubsection{Grandi Valori di $K$}

Quando $K$ è grande (ad esempio, $K$ è una frazione significativa del dataset), 
il modello diventa più rigido. Esso effettua la media su un numero maggiore di punti, 
riducendo la varianza ma aumentando il bias. Questo significa che il modello potrebbe 
non catturare le complessità del dataset e potrebbe risultare in underfitting.

\subsubsection{Scegliere il Valore Ottimale di $K$}

La scelta ottimale di $K$ dipende dal dataset specifico. 
Una tecnica comune per trovare il valore ottimale di $K$ è utilizzare 
la validazione incrociata (cross-validation). In questa tecnica, 
il dataset viene diviso in $k$-folds (sottogruppi), e il modello 
viene addestrato e valutato $k$ volte, ogni volta utilizzando un 
diverso fold come set di validazione e il resto come set di addestramento. 
La media degli errori di validazione per ciascun valore di $K$ viene quindi 
utilizzata per selezionare il valore di $K$ che minimizza l'errore.

\subsection{Interpretazione probabilistica}

In teoria, per effettuare previsioni accurate, sarebbe ideale conoscere la distribuzione 
condizionale dei dati. Tuttavia, nella pratica, questa distribuzione è generalmente sconosciuta, 
rendendo impossibile una stima diretta basata su di essa. Nonostante ciò, metodi come il K-nearest 
neighbors (KNN) riescono comunque a fare previsioni accurate stimando tale distribuzione in maniera non parametrica.

Il KNN stima la distribuzione dei dati basandosi sui \( K \) punti di addestramento più vicini a un punto 
di test \( \hat{\mathbf{x}} \). La probabilità condizionale viene calcolata come la frazione dei punti 
in questo insieme che condividono la stessa caratteristica della variabile di interesse:

\[
Pr(Y = j \mid X = \mathbf{x}_0) = \frac{1}{K} \sum_{i \in N_0} I(y_i = j),
\]

dove \( N_0 \) rappresenta l'insieme dei \( K \) punti di addestramento più vicini a \( \mathbf{x}_0 \) e \( I(y_i = j) \) è una funzione indicatrice che vale 1 se \( y_i \) è uguale a \( j \) e 0 altrimenti.

Nonostante la semplicità del metodo, il KNN può spesso produrre previsioni molto efficaci, avvicinandosi al comportamento ottimale in molti scenari. Tuttavia, la scelta del parametro \( K \) è cruciale: un valore troppo piccolo di \( K \) rende il modello troppo flessibile e sensibile al rumore nei dati, mentre un valore troppo grande può rendere il modello eccessivamente rigido e incapace di catturare la struttura sottostante dei dati.

La relazione tra il tasso di errore di addestramento e quello di test non è sempre diretta. Aumentando la flessibilità del modello (diminuendo \( K \)), il tasso di errore di addestramento tende a diminuire, ma l'errore di test può aumentare se il modello soffre di overfitting. Questo comportamento è ben rappresentato dalla forma a U del grafico dell'errore di test in funzione di \( 1/K \).

La scelta del giusto livello di flessibilità è fondamentale per il successo di qualsiasi metodo di apprendimento statistico. Nel Capitolo 5, torneremo su questo argomento e discuteremo vari metodi per stimare i tassi di errore di test, al fine di scegliere il livello ottimale di flessibilità per un determinato metodo di apprendimento statistico.

\subsection{Comportamento asintotico e convergenza}

\section{Ottimizzazioni}
\label{sec:ottimizzazioni}

\subsection{KD-Tree}
\label{subsec:kd_tree}

\subsection{Ball Tree}
\label{subsec:ball_tree}

\subsection{Hashing}
\label{subsec:hashing}

\subsection{Algoritmi Approximate Nearest Neighbors (ANN)}
\label{subsec:approximate_nearest_neighbors}


\nocite{*}
\bibliographystyle{unsrt}
\bibliography{sample.bib}

\end{document}


\section{Introduzione}

\subsection{Panoramica dell'algoritmo K-Nearest Neighbors (KNN)}
L'algoritmo K-Nearest Neighbors (KNN) rappresenta un pilastro fondamentale 
nell'ambito dell'apprendimento automatico supervisionato, apprezzato per la 
sua semplicità concettuale e la sua efficacia in una vasta gamma di applicazioni. 
La sua filosofia si basa sul principio intuitivo che oggetti simili tendono a 
raggrupparsi nello stesso spazio delle caratteristiche. Questo approccio non 
parametrico permette a KNN di adattarsi a strutture dati complesse e 
a relazioni non lineari, senza fare assunzioni rigide sulla distribuzione dei dati.

\subsection{Funzionamento di KNN}

KNN opera determinando le etichette di classificazione o i valori di 
regressione per un nuovo punto, basandosi sulla vicinanza ai punti di addestramento. 
Il parametro chiave di KNN è $K$, che rappresenta il numero di "vicini" più prossimi 
da considerare durante la fase di predizione. Quando un nuovo dato deve essere 
classificato, KNN calcola la distanza (come vedremo, esistono varie tipologie di distanze) 
tra il dato da classificare 
e tutti i punti di addestramento, quindi seleziona i $K$ punti più vicini. La classe o 
il valore di regressione del nuovo dato è determinato dalla classe maggiormente rappresentata
o dalla media dei valori nei punti vicini, rispettivamente.

\subsection{Potenzialità di KNN}

KNN trova applicazione in numerosi settori grazie alla sua flessibilità e facilità di 
implementazione. Nei sistemi di raccomandazione, ad esempio, può suggerire prodotti o 
contenuti simili a quelli preferiti dall'utente sulla base dei gusti di altri utenti simili 
(vicini). Nell'analisi di immagini e nel riconoscimento di pattern, KNN può classificare 
nuove immagini confrontandole con esempi già noti. In ambito medico, può supportare la diagnosi 
confrontando i sintomi del paziente con casi storici simili.

\subsection{Caratteristiche e Limitazioni}

Una delle caratteristiche distintive di KNN è la sua interpretabilità. Le decisioni di 
classificazione o regressione si basano direttamente sulla vicinanza tra i dati, rendendo 
il processo decisionale trasparente e facilmente comprensibile. Tuttavia, KNN può essere 
sensibile al rumore nei dati e alla presenza di feature non rilevanti, il che può influenzare 
negativamente le previsioni. Inoltre, gestire grandi dataset con KNN può essere computazionalmente 
oneroso, poiché richiede il calcolo delle distanze tra il nuovo dato e tutti i punti di addestramento; 
sopratutto quando si ha a che fare con dataset dimensionalmente complessi.

\subsection{Definizione e concetto di base}

\subsubsection{Dataset, feature e variabile target}

Un dataset in machine learning è una collezione di dati organizzati in un formato strutturato. Ogni riga del dataset rappresenta un'osservazione, mentre ogni colonna rappresenta una caratteristica (feature) o la variabile target (etichetta). Le feature sono attributi che descrivono le osservazioni e possono essere di diversi tipi, come numeriche, categoriche o binarie. La variabile target è ciò che vogliamo predire utilizzando le feature.

Ad esempio, in un dataset per la predizione del prezzo delle case, le feature potrebbero includere la superficie della casa, il numero di camere, la posizione, l'anno di costruzione, ecc. La variabile target sarebbe il prezzo della casa.

\subsubsection{Problema di classificazione}

Consideriamo un esempio reale: la diagnosi precoce di malattie cardiache. Questo è un problema non banale che richiede l'uso del machine learning per essere risolto efficacemente. Il dataset potrebbe includere pazienti con varie caratteristiche cliniche misurate durante esami medici. Le feature potrebbero includere età, genere, pressione sanguigna, livelli di colesterolo, frequenza cardiaca massima, risultati di elettrocardiogrammi, e altre misure cliniche rilevanti. La variabile target sarebbe una variabile binaria che indica la presenza o l'assenza di una malattia cardiaca.

L'algoritmo KNN può essere utilizzato per classificare un nuovo paziente come "a rischio" o "non a rischio" di malattia cardiaca basandosi sui dati storici di altri pazienti. Quando un nuovo paziente entra per una valutazione, KNN calcola le distanze tra le caratteristiche cliniche del nuovo paziente e quelle dei pazienti nel dataset di addestramento. Seleziona i \( K \) pazienti più simili (i vicini più prossimi) e determina la classe del nuovo paziente in base alla maggioranza delle classi dei vicini.

Per esempio, se \( K = 5 \) e tra i 5 pazienti più vicini al nuovo paziente 3 hanno una malattia cardiaca e 2 no, KNN predice che il nuovo paziente è "a rischio" di malattia cardiaca. Questa previsione può aiutare i medici a prendere decisioni informate riguardo ulteriori test o trattamenti, dimostrando l'importanza e l'utilità del Machine Learning in contesti medici critici.

\subsubsection{Problema di regressione}

Un esempio di problema di regressione risolvibile con l'algoritmo K-Nearest Neighbors (KNN) è la previsione del valore di mercato delle proprietà immobiliari in una città. Questo problema richiede un approccio che si affida al machine learning per ottenere stime accurate e affidabili, data la moltitudine di fattori che influenzano i prezzi delle case.

Consideriamo un dataset che include informazioni dettagliate sulle proprietà immobiliari di una città. Le feature possono includere:

\begin{itemize}
    \item Superficie della proprietà (in metri quadrati)
    \item Numero di camere da letto
    \item Numero di bagni
    \item Anno di costruzione
    \item Distanza dai servizi principali (scuole, ospedali, trasporti pubblici)
    \item Valutazioni della qualità del quartiere
    \item Prezzi recenti delle proprietà vicine
\end{itemize}

La variabile target in questo caso è il prezzo di vendita della proprietà.

Quando si vuole stimare il valore di una nuova proprietà, l'algoritmo KNN calcola la distanza tra le caratteristiche della nuova proprietà e quelle delle proprietà nel dataset di addestramento. Utilizzando una metrica di distanza (come la distanza euclidea), KNN identifica i \( K \) immobili più simili. 

Ad esempio, se \( K = 5 \), KNN selezionerà le cinque proprietà più vicine alla nuova proprietà in termini di caratteristiche. Il prezzo stimato per la nuova proprietà sarà la media dei prezzi delle cinque proprietà più vicine.

\[
\hat{y} = \frac{1}{K} \sum_{i \in \mathcal{N}_K(\mathbf{x})} y_i
\]

Dove \( \hat{y} \) è il prezzo stimato della nuova proprietà, \( K \) è il numero di vicini considerati, \( \mathcal{N}_K(\mathbf{x}) \) rappresenta l'insieme dei \( K \) vicini più prossimi e \( y_i \) è il prezzo di una delle proprietà vicine.

Questo approccio di regressione basato su KNN è particolarmente utile perché tiene conto della località spaziale e delle caratteristiche specifiche delle proprietà immobiliari. Inoltre, permette di adattarsi a variazioni non lineari e complesse nei dati, che sono comuni nel mercato immobiliare. La previsione accurata dei prezzi immobiliari è fondamentale per acquirenti, venditori, agenti immobiliari e investitori, rendendo KNN uno strumento prezioso in questo contesto.


\subsection{Sfide e Ottimizzazioni}

La "maledizione della dimensionalità" è una delle sfide principali di KNN, in quanto la 
performance dell'algoritmo può decadere significativamente con l'aumento della dimensionalità 
dei dati. Per mitigare questo problema, sono state sviluppate tecniche come la riduzione della 
dimensionalità e l'uso di strutture dati specializzate (come KD-Trees) per accelerare il calcolo 
delle distanze.

\subsection{Applicazioni Avanzate e Ricerca}

La ricerca attuale su KNN si concentra sulla sua integrazione con tecniche avanzate di 
machine learning, come l'apprendimento semi-supervisionato e il trasferimento di conoscenza, 
per migliorare ulteriormente la sua robustezza e la sua capacità predittiva in scenari complessi.

Concludendo questa introduzione, KNN rimane una scelta potente e versatile per molte applicazioni di machine learning grazie alla sua semplicità, flessibilità e interpretabilità. Nonostante le sfide associate, continua a essere ampiamente utilizzato come punto di partenza per problemi di classificazione e regressione, offrendo risultati affidabili e interpretazioni chiare in vari contesti applicativi.

\subsection{Obiettivi dell'articolo}

L'obiettivo principale di questo articolo è fornire una comprensione completa e dettagliata 
dell'algoritmo K-Nearest Neighbors (KNN) attraverso un'analisi sia teorica che pratica. 
In primo luogo, l'articolo presenterà i fondamenti teorici di KNN, spiegando il funzionamento 
dell'algoritmo, le metriche di distanza utilizzate (come la distanza euclidea e la distanza di Manhattan) 
e l'importanza della scelta del parametro $K$. Verranno, inoltre, discusse le implicazioni di queste scelte 
sulla performance dell'algoritmo.

In secondo luogo, verranno esaminate le proprietà matematiche di KNN, inclusa la complessità 
computazionale e le sfide legate alla "maledizione della dimensionalità". Saranno analizzati i 
trade-off tra bias e varianza per comprendere come ottimizzare le prestazioni di questo algoritmo.

In terzo luogo, l'articolo fornirà un'analisi empirica, confrontando KNN con altri algoritmi 
di machine learning su dataset standard. Questo confronto aiuterà a evidenziare i punti di forza 
e le limitazioni di KNN in scenari pratici.

Infine, l'articolo esplorerà le tecniche di ottimizzazione e miglioramento di KNN, 
come la normalizzazione dei dati, l'uso di KNN pesato e l'implementazione di strutture 
dati efficienti come KD-Trees. 

In sintesi, l'articolo mira a offrire una visione completa e 
dettagliata dell'algoritmo KNN.
\section{Fondamenti Teorici del KNN}
\subsection{Definizione e concetto di base}
\subsection{Definizione matematica formale}
\subsection{Scelta del parametro K}
\subsection{Metriche di distanza}
\subsubsection{Distanza Euclidea}
\subsubsection{Distanza di Manhattan}
\subsubsection{Distanza di Minkowski}
\subsubsection{Altre metriche di distanza}

\section{Proprietà Matematiche e Analisi Teorica}
\subsection{La maledizione della dimensionalità}
\subsection{Complessità computazionale}
\subsection{Trade-off bias-varianza nel KNN}
\subsection{Interpretazione probabilistica del KNN}
\subsection{Comportamento asintotico e convergenza}
\section{Analisi Teorica}
\subsection{La maledizione della dimensionalità}
\subsection{Complessità computazionale}
\subsection{Trade-off bias-varianza nel KNN}
\subsection{Scelta di \( K \)}

Nel contesto del KNN, il (iper)parametro $K$ gioca un ruolo cruciale nel determinare il trade-off tra bias e varianza:

\subsubsection{Piccoli Valori di $K$}

Quando $K$ è piccolo (ad esempio, $K=1$), il modello tende a seguire 
molto da vicino i dati di addestramento. Questo può portare a un basso bias, 
poiché il modello è molto flessibile e può adattarsi alle particolarità dei dati 
di addestramento. Tuttavia, questo porta a una elevata varianza, poiché il modello 
è sensibile al rumore nei dati. In altre parole, un valore di $K$ troppo piccolo può 
causare overfitting.

\subsubsection{Grandi Valori di $K$}

Quando $K$ è grande (ad esempio, $K$ è una frazione significativa del dataset), 
il modello diventa più rigido. Esso effettua la media su un numero maggiore di punti, 
riducendo la varianza ma aumentando il bias. Questo significa che il modello potrebbe 
non catturare le complessità del dataset e potrebbe risultare in underfitting.

\subsubsection{Scegliere il Valore Ottimale di $K$}

La scelta ottimale di $K$ dipende dal dataset specifico. 
Una tecnica comune per trovare il valore ottimale di $K$ è utilizzare 
la validazione incrociata (cross-validation). In questa tecnica, 
il dataset viene diviso in $k$-folds (sottogruppi), e il modello 
viene addestrato e valutato $k$ volte, ogni volta utilizzando un 
diverso fold come set di validazione e il resto come set di addestramento. 
La media degli errori di validazione per ciascun valore di $K$ viene quindi 
utilizzata per selezionare il valore di $K$ che minimizza l'errore.

\subsection{Interpretazione probabilistica}

In teoria, per effettuare previsioni accurate, sarebbe ideale conoscere la distribuzione 
condizionale dei dati. Tuttavia, nella pratica, questa distribuzione è generalmente sconosciuta, 
rendendo impossibile una stima diretta basata su di essa. Nonostante ciò, metodi come il K-nearest 
neighbors (KNN) riescono comunque a fare previsioni accurate stimando tale distribuzione in maniera non parametrica.

Il KNN stima la distribuzione dei dati basandosi sui \( K \) punti di addestramento più vicini a un punto 
di test \( \hat{\mathbf{x}} \). La probabilità condizionale viene calcolata come la frazione dei punti 
in questo insieme che condividono la stessa caratteristica della variabile di interesse:

\[
Pr(Y = j \mid X = \mathbf{x}_0) = \frac{1}{K} \sum_{i \in N_0} I(y_i = j),
\]

dove \( N_0 \) rappresenta l'insieme dei \( K \) punti di addestramento più vicini a \( \mathbf{x}_0 \) e \( I(y_i = j) \) è una funzione indicatrice che vale 1 se \( y_i \) è uguale a \( j \) e 0 altrimenti.

Nonostante la semplicità del metodo, il KNN può spesso produrre previsioni molto efficaci, avvicinandosi al comportamento ottimale in molti scenari. Tuttavia, la scelta del parametro \( K \) è cruciale: un valore troppo piccolo di \( K \) rende il modello troppo flessibile e sensibile al rumore nei dati, mentre un valore troppo grande può rendere il modello eccessivamente rigido e incapace di catturare la struttura sottostante dei dati.

La relazione tra il tasso di errore di addestramento e quello di test non è sempre diretta. Aumentando la flessibilità del modello (diminuendo \( K \)), il tasso di errore di addestramento tende a diminuire, ma l'errore di test può aumentare se il modello soffre di overfitting. Questo comportamento è ben rappresentato dalla forma a U del grafico dell'errore di test in funzione di \( 1/K \).

La scelta del giusto livello di flessibilità è fondamentale per il successo di qualsiasi metodo di apprendimento statistico. Nel Capitolo 5, torneremo su questo argomento e discuteremo vari metodi per stimare i tassi di errore di test, al fine di scegliere il livello ottimale di flessibilità per un determinato metodo di apprendimento statistico.

\subsection{Comportamento asintotico e convergenza}

\section{Ottimizzazioni}
\label{sec:ottimizzazioni}

\subsection{KD-Tree}
\label{subsec:kd_tree}

\subsection{Ball Tree}
\label{subsec:ball_tree}

\subsection{Hashing}
\label{subsec:hashing}

\subsection{Algoritmi Approximate Nearest Neighbors (ANN)}
\label{subsec:approximate_nearest_neighbors}


\nocite{*}
\bibliographystyle{unsrt}
\bibliography{sample.bib}

\end{document}


\section{Introduzione}

\subsection{Panoramica dell'algoritmo K-Nearest Neighbors (KNN)}
L'algoritmo K-Nearest Neighbors (KNN) rappresenta un pilastro fondamentale 
nell'ambito dell'apprendimento automatico supervisionato, apprezzato per la 
sua semplicità concettuale e la sua efficacia in una vasta gamma di applicazioni. 
La sua filosofia si basa sul principio intuitivo che oggetti simili tendono a 
raggrupparsi nello stesso spazio delle caratteristiche. Questo approccio non 
parametrico permette a KNN di adattarsi a strutture dati complesse e 
a relazioni non lineari, senza fare assunzioni rigide sulla distribuzione dei dati.

\subsection{Funzionamento di KNN}

KNN opera determinando le etichette di classificazione o i valori di 
regressione per un nuovo punto, basandosi sulla vicinanza ai punti di addestramento. 
Il parametro chiave di KNN è $K$, che rappresenta il numero di "vicini" più prossimi 
da considerare durante la fase di predizione. Quando un nuovo dato deve essere 
classificato, KNN calcola la distanza (come vedremo, esistono varie tipologie di distanze) 
tra il dato da classificare 
e tutti i punti di addestramento, quindi seleziona i $K$ punti più vicini. La classe o 
il valore di regressione del nuovo dato è determinato dalla classe maggiormente rappresentata
o dalla media dei valori nei punti vicini, rispettivamente.

\subsection{Potenzialità di KNN}

KNN trova applicazione in numerosi settori grazie alla sua flessibilità e facilità di 
implementazione. Nei sistemi di raccomandazione, ad esempio, può suggerire prodotti o 
contenuti simili a quelli preferiti dall'utente sulla base dei gusti di altri utenti simili 
(vicini). Nell'analisi di immagini e nel riconoscimento di pattern, KNN può classificare 
nuove immagini confrontandole con esempi già noti. In ambito medico, può supportare la diagnosi 
confrontando i sintomi del paziente con casi storici simili.

\subsection{Caratteristiche e Limitazioni}

Una delle caratteristiche distintive di KNN è la sua interpretabilità. Le decisioni di 
classificazione o regressione si basano direttamente sulla vicinanza tra i dati, rendendo 
il processo decisionale trasparente e facilmente comprensibile. Tuttavia, KNN può essere 
sensibile al rumore nei dati e alla presenza di feature non rilevanti, il che può influenzare 
negativamente le previsioni. Inoltre, gestire grandi dataset con KNN può essere computazionalmente 
oneroso, poiché richiede il calcolo delle distanze tra il nuovo dato e tutti i punti di addestramento; 
sopratutto quando si ha a che fare con dataset dimensionalmente complessi.

\subsection{Definizione e concetto di base}

\subsubsection{Dataset, feature e variabile target}

Un dataset in machine learning è una collezione di dati organizzati in un formato strutturato. Ogni riga del dataset rappresenta un'osservazione, mentre ogni colonna rappresenta una caratteristica (feature) o la variabile target (etichetta). Le feature sono attributi che descrivono le osservazioni e possono essere di diversi tipi, come numeriche, categoriche o binarie. La variabile target è ciò che vogliamo predire utilizzando le feature.

Ad esempio, in un dataset per la predizione del prezzo delle case, le feature potrebbero includere la superficie della casa, il numero di camere, la posizione, l'anno di costruzione, ecc. La variabile target sarebbe il prezzo della casa.

\subsubsection{Problema di classificazione}

Consideriamo un esempio reale: la diagnosi precoce di malattie cardiache. Questo è un problema non banale che richiede l'uso del machine learning per essere risolto efficacemente. Il dataset potrebbe includere pazienti con varie caratteristiche cliniche misurate durante esami medici. Le feature potrebbero includere età, genere, pressione sanguigna, livelli di colesterolo, frequenza cardiaca massima, risultati di elettrocardiogrammi, e altre misure cliniche rilevanti. La variabile target sarebbe una variabile binaria che indica la presenza o l'assenza di una malattia cardiaca.

L'algoritmo KNN può essere utilizzato per classificare un nuovo paziente come "a rischio" o "non a rischio" di malattia cardiaca basandosi sui dati storici di altri pazienti. Quando un nuovo paziente entra per una valutazione, KNN calcola le distanze tra le caratteristiche cliniche del nuovo paziente e quelle dei pazienti nel dataset di addestramento. Seleziona i \( K \) pazienti più simili (i vicini più prossimi) e determina la classe del nuovo paziente in base alla maggioranza delle classi dei vicini.

Per esempio, se \( K = 5 \) e tra i 5 pazienti più vicini al nuovo paziente 3 hanno una malattia cardiaca e 2 no, KNN predice che il nuovo paziente è "a rischio" di malattia cardiaca. Questa previsione può aiutare i medici a prendere decisioni informate riguardo ulteriori test o trattamenti, dimostrando l'importanza e l'utilità del Machine Learning in contesti medici critici.

\subsubsection{Problema di regressione}

Un esempio di problema di regressione risolvibile con l'algoritmo K-Nearest Neighbors (KNN) è la previsione del valore di mercato delle proprietà immobiliari in una città. Questo problema richiede un approccio che si affida al machine learning per ottenere stime accurate e affidabili, data la moltitudine di fattori che influenzano i prezzi delle case.

Consideriamo un dataset che include informazioni dettagliate sulle proprietà immobiliari di una città. Le feature possono includere:

\begin{itemize}
    \item Superficie della proprietà (in metri quadrati)
    \item Numero di camere da letto
    \item Numero di bagni
    \item Anno di costruzione
    \item Distanza dai servizi principali (scuole, ospedali, trasporti pubblici)
    \item Valutazioni della qualità del quartiere
    \item Prezzi recenti delle proprietà vicine
\end{itemize}

La variabile target in questo caso è il prezzo di vendita della proprietà.

Quando si vuole stimare il valore di una nuova proprietà, l'algoritmo KNN calcola la distanza tra le caratteristiche della nuova proprietà e quelle delle proprietà nel dataset di addestramento. Utilizzando una metrica di distanza (come la distanza euclidea), KNN identifica i \( K \) immobili più simili. 

Ad esempio, se \( K = 5 \), KNN selezionerà le cinque proprietà più vicine alla nuova proprietà in termini di caratteristiche. Il prezzo stimato per la nuova proprietà sarà la media dei prezzi delle cinque proprietà più vicine.

\[
\hat{y} = \frac{1}{K} \sum_{i \in \mathcal{N}_K(\mathbf{x})} y_i
\]

Dove \( \hat{y} \) è il prezzo stimato della nuova proprietà, \( K \) è il numero di vicini considerati, \( \mathcal{N}_K(\mathbf{x}) \) rappresenta l'insieme dei \( K \) vicini più prossimi e \( y_i \) è il prezzo di una delle proprietà vicine.

Questo approccio di regressione basato su KNN è particolarmente utile perché tiene conto della località spaziale e delle caratteristiche specifiche delle proprietà immobiliari. Inoltre, permette di adattarsi a variazioni non lineari e complesse nei dati, che sono comuni nel mercato immobiliare. La previsione accurata dei prezzi immobiliari è fondamentale per acquirenti, venditori, agenti immobiliari e investitori, rendendo KNN uno strumento prezioso in questo contesto.


\subsection{Sfide e Ottimizzazioni}

La "maledizione della dimensionalità" è una delle sfide principali di KNN, in quanto la 
performance dell'algoritmo può decadere significativamente con l'aumento della dimensionalità 
dei dati. Per mitigare questo problema, sono state sviluppate tecniche come la riduzione della 
dimensionalità e l'uso di strutture dati specializzate (come KD-Trees) per accelerare il calcolo 
delle distanze.

\subsection{Applicazioni Avanzate e Ricerca}

La ricerca attuale su KNN si concentra sulla sua integrazione con tecniche avanzate di 
machine learning, come l'apprendimento semi-supervisionato e il trasferimento di conoscenza, 
per migliorare ulteriormente la sua robustezza e la sua capacità predittiva in scenari complessi.

Concludendo questa introduzione, KNN rimane una scelta potente e versatile per molte applicazioni di machine learning grazie alla sua semplicità, flessibilità e interpretabilità. Nonostante le sfide associate, continua a essere ampiamente utilizzato come punto di partenza per problemi di classificazione e regressione, offrendo risultati affidabili e interpretazioni chiare in vari contesti applicativi.

\subsection{Obiettivi dell'articolo}

L'obiettivo principale di questo articolo è fornire una comprensione completa e dettagliata 
dell'algoritmo K-Nearest Neighbors (KNN) attraverso un'analisi sia teorica che pratica. 
In primo luogo, l'articolo presenterà i fondamenti teorici di KNN, spiegando il funzionamento 
dell'algoritmo, le metriche di distanza utilizzate (come la distanza euclidea e la distanza di Manhattan) 
e l'importanza della scelta del parametro $K$. Verranno, inoltre, discusse le implicazioni di queste scelte 
sulla performance dell'algoritmo.

In secondo luogo, verranno esaminate le proprietà matematiche di KNN, inclusa la complessità 
computazionale e le sfide legate alla "maledizione della dimensionalità". Saranno analizzati i 
trade-off tra bias e varianza per comprendere come ottimizzare le prestazioni di questo algoritmo.

In terzo luogo, l'articolo fornirà un'analisi empirica, confrontando KNN con altri algoritmi 
di machine learning su dataset standard. Questo confronto aiuterà a evidenziare i punti di forza 
e le limitazioni di KNN in scenari pratici.

Infine, l'articolo esplorerà le tecniche di ottimizzazione e miglioramento di KNN, 
come la normalizzazione dei dati, l'uso di KNN pesato e l'implementazione di strutture 
dati efficienti come KD-Trees. 

In sintesi, l'articolo mira a offrire una visione completa e 
dettagliata dell'algoritmo KNN.
\section{Fondamenti Teorici del KNN}
\subsection{Definizione e concetto di base}
\subsection{Definizione matematica formale}
\subsection{Scelta del parametro K}
\subsection{Metriche di distanza}
\subsubsection{Distanza Euclidea}
\subsubsection{Distanza di Manhattan}
\subsubsection{Distanza di Minkowski}
\subsubsection{Altre metriche di distanza}

\section{Proprietà Matematiche e Analisi Teorica}
\subsection{La maledizione della dimensionalità}
\subsection{Complessità computazionale}
\subsection{Trade-off bias-varianza nel KNN}
\subsection{Interpretazione probabilistica del KNN}
\subsection{Comportamento asintotico e convergenza}
\section{Analisi Teorica}
\subsection{La maledizione della dimensionalità}
\subsection{Complessità computazionale}
\subsection{Trade-off bias-varianza nel KNN}
\subsection{Scelta di \( K \)}

Nel contesto del KNN, il (iper)parametro $K$ gioca un ruolo cruciale nel determinare il trade-off tra bias e varianza:

\subsubsection{Piccoli Valori di $K$}

Quando $K$ è piccolo (ad esempio, $K=1$), il modello tende a seguire 
molto da vicino i dati di addestramento. Questo può portare a un basso bias, 
poiché il modello è molto flessibile e può adattarsi alle particolarità dei dati 
di addestramento. Tuttavia, questo porta a una elevata varianza, poiché il modello 
è sensibile al rumore nei dati. In altre parole, un valore di $K$ troppo piccolo può 
causare overfitting.

\subsubsection{Grandi Valori di $K$}

Quando $K$ è grande (ad esempio, $K$ è una frazione significativa del dataset), 
il modello diventa più rigido. Esso effettua la media su un numero maggiore di punti, 
riducendo la varianza ma aumentando il bias. Questo significa che il modello potrebbe 
non catturare le complessità del dataset e potrebbe risultare in underfitting.

\subsubsection{Scegliere il Valore Ottimale di $K$}

La scelta ottimale di $K$ dipende dal dataset specifico. 
Una tecnica comune per trovare il valore ottimale di $K$ è utilizzare 
la validazione incrociata (cross-validation). In questa tecnica, 
il dataset viene diviso in $k$-folds (sottogruppi), e il modello 
viene addestrato e valutato $k$ volte, ogni volta utilizzando un 
diverso fold come set di validazione e il resto come set di addestramento. 
La media degli errori di validazione per ciascun valore di $K$ viene quindi 
utilizzata per selezionare il valore di $K$ che minimizza l'errore.

\subsection{Interpretazione probabilistica}

In teoria, per effettuare previsioni accurate, sarebbe ideale conoscere la distribuzione 
condizionale dei dati. Tuttavia, nella pratica, questa distribuzione è generalmente sconosciuta, 
rendendo impossibile una stima diretta basata su di essa. Nonostante ciò, metodi come il K-nearest 
neighbors (KNN) riescono comunque a fare previsioni accurate stimando tale distribuzione in maniera non parametrica.

Il KNN stima la distribuzione dei dati basandosi sui \( K \) punti di addestramento più vicini a un punto 
di test \( \hat{\mathbf{x}} \). La probabilità condizionale viene calcolata come la frazione dei punti 
in questo insieme che condividono la stessa caratteristica della variabile di interesse:

\[
Pr(Y = j \mid X = \mathbf{x}_0) = \frac{1}{K} \sum_{i \in N_0} I(y_i = j),
\]

dove \( N_0 \) rappresenta l'insieme dei \( K \) punti di addestramento più vicini a \( \mathbf{x}_0 \) e \( I(y_i = j) \) è una funzione indicatrice che vale 1 se \( y_i \) è uguale a \( j \) e 0 altrimenti.

Nonostante la semplicità del metodo, il KNN può spesso produrre previsioni molto efficaci, avvicinandosi al comportamento ottimale in molti scenari. Tuttavia, la scelta del parametro \( K \) è cruciale: un valore troppo piccolo di \( K \) rende il modello troppo flessibile e sensibile al rumore nei dati, mentre un valore troppo grande può rendere il modello eccessivamente rigido e incapace di catturare la struttura sottostante dei dati.

La relazione tra il tasso di errore di addestramento e quello di test non è sempre diretta. Aumentando la flessibilità del modello (diminuendo \( K \)), il tasso di errore di addestramento tende a diminuire, ma l'errore di test può aumentare se il modello soffre di overfitting. Questo comportamento è ben rappresentato dalla forma a U del grafico dell'errore di test in funzione di \( 1/K \).

La scelta del giusto livello di flessibilità è fondamentale per il successo di qualsiasi metodo di apprendimento statistico. Nel Capitolo 5, torneremo su questo argomento e discuteremo vari metodi per stimare i tassi di errore di test, al fine di scegliere il livello ottimale di flessibilità per un determinato metodo di apprendimento statistico.

\subsection{Comportamento asintotico e convergenza}

\section{Ottimizzazioni}
\label{sec:ottimizzazioni}

\subsection{KD-Tree}
\label{subsec:kd_tree}

\subsection{Ball Tree}
\label{subsec:ball_tree}

\subsection{Hashing}
\label{subsec:hashing}

\subsection{Algoritmi Approximate Nearest Neighbors (ANN)}
\label{subsec:approximate_nearest_neighbors}


\nocite{*}
\bibliographystyle{unsrt}
\bibliography{sample.bib}

\end{document}