\section{Objects}
In the realm of mathematics, fundamental objects like scalars, vectors, matrices, and tensors serve as the building blocks for expressing a wide range of mathematical and scientific concepts. These entities encapsulate different levels of complexity, from singular values to multi-dimensional structures, each offering unique properties and applications.

\subsection{Scalars}
Scalars are elements of a field, like $\mathbb{R}$,that represent quantities that only have magnitude, such as temperature, mass, distance, or time. Scalars have zero dimensions. They are considered point values in a mathematical space.

Since a scalar has no dimensions, its shape is represented as an empty set.

Scalars are typically denoted by lowercase letters, e.g., $a,b,c$ and they are treated as constants; they can be real numbers, complex numbers, or elements from other mathematical fields.
\\

For example 

$$
a \in \mathbb{F}
$$

where is a scalar in a field $\mathbb{F}$.

\subsection{Vectors}
A vector is an ordered collection of elements of a field that represents a quantity with both magnitude and direction. Vectors are used to describe various physical quantities, such as displacement, velocity, and force.
In a geometric sense, they can be visualized as directed line segments with a specific length and direction in space.

Vectors have one dimension, so the shape of a vector is a 1-dimensional tuple $(n)$, where $n$ is the number of elements in the vector, also known as vector size. 
\\

Here are a few key points to understand about vectors:
\begin{itemize}
    \item \textbf{Magnitude:} This is the length of the vector.

    \item \textbf{Direction:} This indicates the way the vector points.

    \item \textbf{Components:} The individual parts that make up a vector.
\end{itemize}

They are typically denoted by a letter with an arrow above. For example

$$
\vec v =\begin{bmatrix}
    v_1\\
    v_2\\
    \vdots\\
    v_n
\end{bmatrix} \in \mathbb{F}^n,
$$

where $\vec v$ is a vector in $\mathbb F^n$, $(n)$ is the vector shape, $n$ is the vector size and $v_1, v_2, \dots, v_n \in \mathbb{F}$ are vector components.

If we wish to access the component within the vectors we can employ the following mathematical notation: 

$$
\vec v^{\ (1)} = v_1
$$

And to access all elements of a dimension we can use ":" instead of numerical indexes:

$$
\vec v^{\ (:)} = \vec v
$$

\subsection{Matrices}
A matrix is a two-dimensional array of elements of a field, organized in rows and columns. It is used to represent and manipulate data in various applications, including linear transformations and systems of equations.

The shape of a matrix is a 2-dimensional tuple $(m,n)$, where $m$ is the number of rows and $n$ is the number of columns. 

Matrices are typically denoted by uppercase letters, e.g., $A, B, C$. For example 

$$
A_{m,n} = \begin{bmatrix} 
    a_{11} & \dots  & a_{1n}\\
    \vdots & \ddots & \vdots\\
    a_{m1} & \dots  & a_{mn} 
\end{bmatrix} \in \mathbb{F}^{m \times n}
$$

where $A_{m,n}$ is a matrix in $\mathbb{F}^{m \times n}$, $(m, n)$ is the matrix shape and $a_{i,j} \in \mathbb F$.

If we wish to access the component within the matrix we can employ the following mathematical notation: 

$$
A_{m,n}^{(i, j)} = a_{i,j}
$$

And to access all elements of a dimension we can use ":" instead of numerical indexes:

$$
A_{m,n}^{(i, :)} = \begin{bmatrix}
    a_{i,1}\\
    a_{i, 2}\\
    \vdots\\
    a_{i, n}
\end{bmatrix}, \quad A_{m,n}^{(:, j)} = \begin{bmatrix}
    a_{1,j}\\
    a_{2,j}\\
    \vdots\\
    a_{m,j}
\end{bmatrix}, \quad  A_{m,n}^{(:, :)} = A_{m,n}
$$


\subsubsection{Square Matrices}

A square matrix $A_{m,n}$ is a matrix with the same number of rows and columns, i.e., $m = n$.

\subsubsection{Triangular Matrices}
A lower triangular matrix $A$ is a square matrix where all entries above the main diagonal are zero, i.e., $a_{i,j} = 0$ for $i < j$.

An upper triangular matrix $A$ is a square matrix where all entries below the main diagonal are zero, i.e., $a_{i,j} = 0$ for $i > j$.

\subsubsection{Diagonal Matrices}

A diagonal matrix $D$ is a square matrix where all entries outside the main diagonal are zero, i.e., $a_{i,j} = 0$ for $i \neq j$.

\subsubsection{Symmetric Matrices}
A symmetric matrix $A_{n,n}$ is a square matrix where the elements are symmetric with respect to the main diagonal. 
In other words, if $a_{i,j}$ is an entry in the matrix at the $i$-th row and $j$-th column, then $a_{i,j} = a_{j,i}$ 
for all $i$ and $j$ in the matrix. This symmetry implies that the values on one side of the main diagonal mirror the 
values on the other side.

% Diagonale di una matrice
%%Null matrix