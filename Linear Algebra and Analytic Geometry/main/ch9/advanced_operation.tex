\subsection{Norm}
The norm of an algebraic object is defined as a function that assigns to each element of the algebraic structure a non-negative real number, representing its length or magnitude, satisfying certain properties. The norm function is defined as $||\cdot||_p: V \rightarrow \mathbb F$, from a vector space $V$ over a field $\mathbb F$ to the field itself.

For a generic vector $\vec v $ it is defined as:

$$
||\vec v||_p = \sqrt[p]{v_1^p + \dots + v_n^p}
$$

\textbf{Three Properties of Norms:}

Let $\mathbf{v}$ be a vector in a vector space, and let $||\mathbf{v}||$ denote its norm. Then, the norm $||\cdot||$ satisfies the following properties:

\begin{enumerate}
    \item \textbf{Non-negativity:} $||\mathbf{v}|| \geq 0$, with equality if and only if $\mathbf{v} = \mathbf{0}$.
    
    \item \textbf{Scalar Multiplication:} For any scalar $\alpha$, $||\alpha \mathbf{v}|| = |\alpha| \cdot ||\mathbf{v}||$.
    
    \item \textbf{Triangle Inequality:} $||\mathbf{u} + \mathbf{v}|| \leq ||\mathbf{u}|| + ||\mathbf{v}||$.
\end{enumerate}