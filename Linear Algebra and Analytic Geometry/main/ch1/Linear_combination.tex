%%%%%%%%%%%%%%%%%%%%%%%%%%%%%%%%%%%%%%%%LINEAR-COMBINATIONS%%%%%%%%%%%%%%%%%%%%%%%%%%%%%%%%%%%%%%%%%%%%%%%%%%%%%
\section{Linear Combinations}

In the realm of vector spaces, the concept of a \emph{linear combination} holds significant importance. A \emph{linear combination} involves combining vectors in a space by scaling each vector by a scalar and then summing them up. 


    Let $V$ be a vector space over the field $\mathbb F$. If $\vec{v}_1, . \ . \ ., \vec v_s \in V$ are vectors and $a_1, . \ . \ ., a_s \in \mathbb F$ are scalars, then the \textbf{linear combination} of those vectors with those scalars as coefficients is 

$$
a_1 \vec v_1 + \dots + a_s \vec v_s \ \in V, \quad a_1, . \ . \ ., a_s \in \mathbb F.
$$

\label{def:linear-combination}

Understanding linear combinations is foundational in linear algebra, providing insights into the structure and properties of vector spaces. Linear combinations play a crucial role in defining concepts such as span, linear independence, and solutions to systems of linear equations.
%%%%%%%%%%%%%%%%%%%%%%%%%%%%%%%%%%%%%%%%%%SPAN%%%%%%%%%%%%%%%%%%%%%%%%%%%%%%%%%%%%%%%%%%%%%%%%%
\subsection{Span}

The \textbf{span} of a nonempty subset $S$ of a vector space $V$ over a field $\mathbb F$, is the set of all possible linear combinations of those vectors.

$$
\textit{span}(\vec{v}_1, \vec{v}_2, \ldots, \vec{v}_k) = \left\{ \sum_{i=1}^k \lambda_i \vec{v}_i \ \middle| \ k \in \mathbb{N}, \vec{v}_i \in V, \lambda_i \in \mathbb F \right\}
$$

The span of the empty subset of a vector space is its trivial subspace.
\\

In simpler terms, the span is the set of all possible vectors that can be formed by linear combinations of the given vectors.
\\

Let $S \subseteq V$. The set $S$ is a \textbf{spanning set} of $V$ if \textit{every} $\vec v \in V$ can be written as linear combination of vectors in $S$.


\subsubsection{The Span of a Subset of a Vector Space is a Vector Subspace}

Let $S$ be a set of vectors of vector space $V$. The {\textit{span($S$)}} is a vector subspace.
\\

It is easy to show that the span of $S$ contain the $\vec 0$ and is closed under addition and scalar multiplication.

All the other axioms are inherited from $V$.
\\

The span plays a crucial role in determining the extent or reach of a set of vectors. If the span of a set of vectors is the entire vector space $V$, then the vectors are said to \emph{span} the space.