\section{Intuitive Approach to Basis Change}

Let $B = \{\vec{v}_1, \vec{v}_2, \ldots, \vec{v}_n\}$ and $B' = \{\vec{u}_1, \vec{u}_2, \ldots, \vec{u}_n\}$ be two bases for a vector space $V$.

Compute the coordinates of a vector with respect to a basis is a way to represent a vector as a linear combination of the basis vectors, finding the coefficient to apply to each basis vector.

We can find a way to "translate" the coordinates of a vector $\vec{x}$ from a basis $B$ to a basis $B'$ (holding the same vector). Suppose $\vec{x}$ has coordinates $[\vec x]_B$ with respect to $B$ and coordinates $[\vec x]_{B'}$ with respect to $B'$. Then, there exists a matrix $P$, such that:

\[
[ \ \vec x \ ]_{B'} = P[\ \vec x \ ]_B
\]

where $P$ is the transition matrix from $B$ to $B'$.

To find out how the vector coordinates change from a basis to an other, we must know the coordinates of the basis $B$ vectors with respect to $B'$ basis.
\\

Let $\vec x = d_1 \vec v_1 + d_2 \vec v_2 + \ldots + d_n \vec v_n$ be an arbitrary vector in $V$. The coordinate vector of $\vec x$ with respect to the basis $B$ is

\[
[\ \vec x \ ]_B = \begin{bmatrix}
    d_1 \\
    d_2 \\
    \vdots \\
    d_n
\end{bmatrix}.
\]

If

\[
\begin{aligned}
    \vec v_1 &= c_{1,1} \vec u_1 + c_{2,1} \vec u_2 + \ldots + c_{n,1} \vec u_n \\
    \vec v_2 &= c_{1,2} \vec u_1 + c_{2,2} \vec u_2 + \ldots + c_{n,2} \vec u_n \\
    & \vdots \\
    \vec v_n &= c_{1,n} \vec u_1 + c_{2,n} \vec u_2 + \ldots + c_{n,n} \vec u_n,
\end{aligned}
\]

then you can write

\[
\begin{aligned}
    \vec x &= d_1 \vec v_1 + d_2 \vec v_2 + \ldots + d_n \vec v_n \\
    &= d_1(c_{1,1} \vec u_1 + \ldots + c_{n,1} \vec u_n) + \ldots + d_n(c_{1,n} \vec u_1 + \ldots + c_{n,n} \vec u_n) \\
    &= (d_1 c_{1,1} + \ldots + d_n c_{1,n}) \vec u_1 + \ldots + (d_1 c_{n,1} + \ldots + d_n c_{n,n}) \vec u_n,
\end{aligned}
\]

which implies

\[
[ \ \vec x \ ]_{B'} =\begin{bmatrix}
    c_{1,1}d_1 + c_{12,}d_2 + \cdots + c_{1,n}d_n \\
    c_{2,1}d_1 + c_{2,2}d_2 + \cdots + c_{2,n}d_n \\
    \vdots \\
    c_{n,1}d_1 + c_{n,2}d_2 + \cdots + c_{n,n}d_n
\end{bmatrix} = \begin{bmatrix}
    c_{1,1} & c_{1,2} & \cdots & c_{1,n} \\
    c_{2,1} & c_{2,2} & \cdots & c_{2,n} \\
    \vdots & \vdots & \ddots & \vdots \\
    c_{n,1} & c_{n,2} & \cdots & c_{n,n}
\end{bmatrix} \begin{bmatrix}
    d_1 \\
    d_2 \\
    \vdots \\
    d_n
\end{bmatrix} = P[ \ \vec x \ ]_B.
\]

So, the transition matrix from $B$ to $B'$ is

\[
P = \begin{bmatrix}
    c_{1,1} & c_{1,2} & \cdots & c_{1,n} \\
    c_{2,1} & c_{2,2} & \cdots & c_{2,n} \\
    \vdots & \vdots & \ddots & \vdots \\
    c_{n,1} & c_{n,2} & \cdots & c_{n,n}
\end{bmatrix}.
\]

In the section \ref{subsection:indepth-basis-change} of the next chapter, we introduce a general method to compute the transition matrices between two coordinate systems of a vector space.